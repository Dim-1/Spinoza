\documentclass[a4paper,12pt,notitlepage]{report}
\usepackage[T1]{fontenc}
\usepackage[utf8]{inputenc}
\usepackage[italian]{babel}
\raggedbottom


\begin{document}
	\author{Sandro Della Maggiiore}
	\title{Spinoza}
	\date{Novembre 2023}
	\maketitle
	\chapter*{Introduzione}
	
	In questo breve scritto voglio  raccontare del filosofo che mi ha colpito di più da quando mi sono avvicinato alla filosofia: Spinoza.
	
	Per arrivare fino a qua, ho svolto un percorso storico che sta andando avanti, quindi forse incontrerò  altri pensatori che mi affascineranno altrettanto quanto Spinoza. Sta di fatto che Barruch mi è entrato nella testa più di Platone per la bellezza degli scritti, più degli stoici per il loro modo di vedere la vita (e a cui Spinoza gli è un po' debitore), più di Plotino per il fascino complicato della sua dottrina.
	
	Qui riassumerò la filosofia di Spinoza, e mi soffermerò sugli aspetti che  personalmente più mi hanno colpito: quelli che si adattano alla vita nostra contemporanea, che sono veramente tanti; sopratutto la sua umanità e il voler dimostrare a tutti i costi che tutti gli uomini possono raggiungere la massima felicità, se vivono insieme e non uno contro l'altro. \textbf{Spinoza è veramente un filosofo attuale, fresco a quasi 350 anni dalla morte.}
	\chapter*{Vita}
	
	
	Barruch Spinoza nacque ad Amsterdam il 24 novembre 1632
	
	
	
\end{document}