\chapter{Trattato teologico politico}

\bigskip
\bigskip

\begin{quotation}
	\small Si tratta di un testo senza Dio, da mettere al bando in tutti i paesi.
	
	Un testo empio, scritto con astuzia diabolica, meglio sarebbe seppellirlo per sempre nell'eterno oblio.
	
	Si tratta di un libro pieno di deliberati abomini, che ogni persona ragionevole, e quindi ogni cristiano, dovrebbe abolire.
	
	Responsabile dell'aumento di atti peccaminosi che fanno levare grida al cielo, di ingiuria a nome di Dio, empietà, maledizioni, profanazioni...
	
	\textbf{Un libro forgiato all'inferno dall'ebreo apostata, a quattro mani con il Diavolo.}\footnote{Alcuni commenti del XVII secolo al trattato teologico politico.}
		
\end{quotation}

I libri scritti in reazione al trattato teologico politico di Spinoza potrebbero riempire biblioteche intere, tanto fu il clamore suscitato all'uscita di questo testo.


\newpage