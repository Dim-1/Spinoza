\chapter{Trattato teologico politico}

\bigskip
\bigskip

\begin{quotation}
	\small Si tratta di un testo senza Dio, da mettere al bando in tutti i paesi.
	
	Un testo empio, scritto con astuzia diabolica, meglio sarebbe seppellirlo per sempre nell'eterno oblio.
	
	Si tratta di un libro pieno di deliberati abomini, che ogni persona ragionevole, e quindi ogni cristiano, dovrebbe abolire.
	
	Responsabile dell'aumento di atti peccaminosi che fanno levare grida al cielo, di ingiuria a nome di Dio, empietà, maledizioni, profanazioni...
	
	\textbf{Un libro forgiato all'inferno dall'ebreo apostata, a quattro mani con il Diavolo.}\footnote{Alcuni commenti del XVII secolo al trattato teologico politico.}
		
\end{quotation}

I libri scritti in reazione al "Trattato teologico politico" (che abbreviamo con la sigla: Ttp) di Spinoza potrebbero riempire biblioteche intere, tanto fu il clamore suscitato all'uscita di questo testo.

L'opera fu iniziata nel 1665, utilizzando dottrine già esposte nell'Etica (che era composta fino a metà della quarta parte), dove sono dimostrate con maggior rigore; è' pubblicata anonima nel 1670. 

L'intento è difendere la libertà di pensiero, soppressa sopratutto dall'eccessiva autorità dei predicatori, che hanno come ulteriore colpa quella di impedire agli uomini di applicare il loro intelletto. Secondo Spinoza, gli autori della Bibbia intendevano istruire il popolo facendo ricorso all'esperienza e alle parole, e non al ragionamento rigoroso. Le sacre scritture insegnano la morale, ovvero in che modo l'uomo deve agire per raggiungere felicità e salvezza. 

Le religioni dovrebbero rinunciare alle pretese conoscitive e recuperare la funzione etica, per non ostacolare una pacifica e civile convivenza dell'umanità. Per Spinoza la beatitudine consiste nell'amore intellettuale di Dio, cioè nella conoscenza della natura, che ci porta per vie razionali a cercare il bene (utile) dell'umanità intera. La rivelazione deve servire per giungere alla felicità e alla beatitudine senza la conoscenza, indicando il miglior comportamento, praticando amore verso il prossimo. Ragione e fede devono portarci entrambe alla salvezza.

Nella prefazione del Trattato, Spinoza spiega l'origine della superstizione, la tendenza degli uomini a credere a qualsiasi cosa, insane fantasie e immagini di fantasmi inesistenti: gli uomini cadono nella superstizione "per il raggiungimento delle vanità che essi
desiderano", ogni volta che tale raggiungimento appaia impedito. Infatti la prefazione si
apre con la seguente considerazione: "Se gli uomini potessero procedere a ragion veduta in tutte le
loro cose o se la fortuna fosse loro sempre propizia, non andrebbero soggetti ad alcuna
superstizione". Quando infatti le cose vanno bene, allora gli uomini sono pieni "di vanità e di
presunzione" e non sentono il bisogno di ascoltare consigli. Quando invece vanno
male, perdono ogni fiducia nella ragione, tanto che si affidano a qualsivoglia
superstiziosa fantasticheria, pur di placare il timore e alimentare la speranza, le stesse passioni che spingono i singoli ad aggregarsi in uno stato civile rinunciando alle proprie libertà, come visto nella quarta parte dell'Etica. Quindi la superstizione nasce ed è mantenuta dalla paura, dalla speranza, l'odio, l'ira e l'inganno, tutte affezioni fortissime. La superstizione è generata da passioni a loro volta dovute all'ignoranza umana verso se stessi e la dinamica dei desideri.

La superstizione può essere facilmente utilizzata come mezzo di controllo, oggi come nel '600: le religioni (o altri enti) la sfruttano per dirigere la massa secondo gli interessi di potere; si rende necessario un'opera oppressiva di regolamentazione della pratica religiosa (o civile) e di indottrinamento, per impedire dissensi e deviazioni dovuti proprio alla volubilità del popolo superstizioso. Capiamo adesso come la libertà di pensare sia importante per Spinoza, per conservare la pace dello stato e impedire la degenerazione della religione.

Ciò che Spinoza ha in mente è
una grandiosa rivoluzione politica il cui fine ultimo è la liberazione etica dell'uomo, di riscattare gli uomini dalla paura
della morte e dalla vanità del desiderio, dal timore superstizioso degli Dei e dall'ignoranza circa la
reale natura delle cose, inclusa la natura umana. Spinoza dice esplicitamente a chi è rivolto il Tts: è scritto per il "lettore filosofo", per chi vuole conoscere e non per chi desidera desiderare le sue vanità e continuativamente schiavo della superstizione religiosa. La forma  discorsiva di questa opera la rese fruibile ad un pubblico ampio ed eterogeneo, causando a Spinoza i noti problemi di cui le citazioni iniziali sono un piccolo esempio.

Il metodo che Spinoza adotta per liberare la mente umana e indirizzarla verso l'uso della ragione, è applicare alla lettura della Bibbia il metodo razionale
della scienza: evitare ogni forma di pregiudizio, procedere con ragionamenti chiari, stare ai fatti (ciò che
letteralmente è detto nel testo) e cercarne le cause reali e non immaginarie.









\newpage