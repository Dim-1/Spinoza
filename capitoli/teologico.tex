\chapter{Trattato teologico politico}

\bigskip
\bigskip

\begin{quotation}
	\small Si tratta di un testo senza Dio, da mettere al bando in tutti i paesi.
	
	Un testo empio, scritto con astuzia diabolica, meglio sarebbe seppellirlo per sempre nell'eterno oblio.
	
	Si tratta di un libro pieno di deliberati abomini, che ogni persona ragionevole, e quindi ogni cristiano, dovrebbe abolire.
	
	Responsabile dell'aumento di atti peccaminosi che fanno levare grida al cielo, di ingiuria a nome di Dio, empietà, maledizioni, profanazioni...
	
	\textbf{Un libro forgiato all'inferno dall'ebreo apostata, a quattro mani con il Diavolo.}\footnote{Alcuni commenti del XVII secolo al trattato teologico politico.}
		
\end{quotation}

I libri scritti in reazione al Trattato teologico politico di Spinoza potrebbero riempire biblioteche intere, tanto fu il clamore suscitato all'uscita di questo testo.

L'opera fu iniziata nel 1665, utilizzando dottrine già esposte nell'Etica (che era composta fino a metà della quarta parte), dove sono dimostrate con maggior rigore; è' pubblicata anonima nel 1670. 

L'intento è di difendere la libertà di pensiero, soppressa sopratutto dall'eccessiva autorità dei predicatori, che hanno come ulteriore colpa quella di impedire agli uomini di applicare il loro intelletto. Secondo Spinoza, gli autori della Bibbia intendevano istruire il popolo facendo ricorso all'esperienza e alle parole, e non al ragionamento rigoroso. Le sacre scritture insegnano la morale, ovvero in che modo l'uomo deve agire per raggiungere felicità e salvezza. 

Le religioni dovrebbero rinunciare alle pretese conoscitive e recuperare la funzione etica, per non ostacolare una pacifica e civile convivenza dell'umanità. Per Spinoza la beatitudine consiste nell'amore intellettuale di Dio, cioè nella conoscenza della natura, che ci porta per vie razionali a cercare il bene (utile) dell'umanità intera. La rivelazione deve servire per giungere alla felicità e alla beatitudine senza la conoscenza, praticando amore verso il prossimo. Ragione e fede devono portarci entrambe alla salvezza.

Nella prefazione del Trattato, Spinoza spiega l'origine della superstizione, la tendenza degli uomini a credere a qualsiasi cosa: essa nasce ed è mantenuta dalla paura, dalla speranza, l'odio, l'ira e l'inganno, che sono affezioni fortissime.

\newpage