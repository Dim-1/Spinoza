\chapter{Trattato teologico politico}

\bigskip
\bigskip

\begin{quotation}
	\small Si tratta di un testo senza Dio, da mettere al bando in tutti i paesi.
	
	Un testo empio, scritto con astuzia diabolica, meglio sarebbe seppellirlo per sempre nell'eterno oblio.
	
	Si tratta di un libro pieno di deliberati abomini, che ogni persona ragionevole, e quindi ogni cristiano, dovrebbe abolire.
	
	Responsabile dell'aumento di atti peccaminosi che fanno levare grida al cielo, di ingiuria a nome di Dio, empietà, maledizioni, profanazioni \dots
	
	\textbf{Un libro forgiato all'inferno dall'ebreo apostata, a quattro mani con il Diavolo.}\footnote{Alcuni commenti del XVII secolo al trattato teologico politico.}
		
\end{quotation}

I libri scritti in reazione al "Trattato teologico politico" (che abbreviamo con la sigla: Ttp) di Spinoza potrebbero riempire biblioteche intere, tanto fu il clamore suscitato all'uscita di questo testo.

L'opera fu iniziata nel 1665, utilizzando dottrine già esposte nell'Etica (che era composta fino a metà della quarta parte), dove sono dimostrate con maggior rigore; è' pubblicata anonima nel 1670. 

Come il titolo lascia capire, il Ttp è diviso in due parti, quella religiosa dei quindici capitoli iniziali e quella politica dei cinque finali.

\section[Parte religiosa]{Parte religiosa del Ttp}

L'intento del Ttp è difendere la libertà di pensiero, soppressa dall'eccessiva autorità dei predicatori, che hanno come ulteriore colpa quella di impedire agli uomini di applicare il loro intelletto. Secondo Spinoza, gli autori della Bibbia intendevano istruire il popolo facendo ricorso all'esperienza pratica e ai racconti, e non al ragionamento rigoroso. Le sacre scritture insegnano la morale, ovvero in che modo l'uomo deve agire per raggiungere felicità e salvezza. 

Le religioni dovrebbero rinunciare alle pretese conoscitive e recuperare la funzione etica, per non ostacolare una pacifica e civile convivenza dell'umanità. Per Spinoza la beatitudine consiste nell'amore intellettuale di Dio, cioè nella conoscenza della natura, che ci porta per vie razionali a cercare il bene (utile) dell'umanità intera. La rivelazione deve servire per giungere alla felicità e alla beatitudine senza la conoscenza, indicando il miglior comportamento, praticando amore verso il prossimo. Quindi ragione e fede hanno lo stesso scopo, devono portarci entrambe alla salvezza.

Nella prefazione del Trattato, Spinoza spiega l'origine della \textbf{superstizione, la tendenza umana a credere a qualsiasi cosa}, insane fantasie e immagini di fantasmi inesistenti: \textbf{gli uomini cadono nella superstizione "per il raggiungimento delle vanità che essi
desiderano"}\textbf{, ogni volta che tale raggiungimento appaia difficoltoso}. Infatti la prefazione si
apre con la seguente considerazione: 

\begin{quotation}
	\small Se gli uomini potessero procedere a ragion veduta in tutte le
	loro cose o se la fortuna fosse loro sempre propizia, non andrebbero soggetti ad alcuna
	superstizione.\footnote{Ttp, prefazione, a cura di Alessandro Dini, Bompiani, 2019.}
\end{quotation}

Quando le cose vanno bene, allora gli uomini sono pieni "di vanità e di
presunzione" e non sentono il bisogno di ascoltare consigli. Quando invece vanno
male, perdono ogni fiducia nella ragione, tanto che si affidano a qualsiasi
superstiziosa fantasticheria, pur di placare il timore e alimentare la speranza, le stesse passioni che spingono i singoli ad aggregarsi in uno stato civile rinunciando alle proprie libertà, come visto nella quarta parte dell'Etica. Quindi la superstizione nasce ed è mantenuta dalla paura, dalla speranza, l'odio, l'ira e l'inganno, tutte affezioni fortissime, dovute all'ignoranza umana verso se stessi e la dinamica dei desideri.

La superstizione può essere facilmente utilizzata come mezzo di controllo, oggi come nel '600: le religioni (o altre organizzazioni) la sfruttano per dirigere la massa secondo gli interessi di potere; si rende poi necessario un'opera oppressiva di regolamentazione della pratica religiosa (o civile) e di indottrinamento, per impedire dissensi e deviazioni dovuti proprio alla volubilità del popolo superstizioso. Capiamo adesso come la libertà di pensare sia importante per Spinoza, per conservare la pace dello stato e impedire la degenerazione della religione.

Ciò che Spinoza ha in mente è
una grandiosa rivoluzione politica il cui fine ultimo è la liberazione etica dell'uomo, di riscattare le persone dalla paura
della morte e dalla vanità del desiderio, dal timore superstizioso degli Dei e dall'ignoranza circa la
reale natura delle cose, inclusa la natura umana. Spinoza dice esplicitamente \textbf{a chi è rivolto il Tts}: è scritto per il "lettore filosofo", \textbf{per chi vuole conoscere e non per chi desidera desiderare le sue vanità} e rimanere schiavo della superstizione religiosa. La forma  discorsiva di questa opera la rese fruibile ad un pubblico ampio ed eterogeneo, causando a Spinoza i noti problemi di cui le citazioni iniziali sono un piccolo esempio.

Il metodo che Spinoza adotta per liberare la mente umana dalla superstizione, è applicare alla lettura della Bibbia il metodo razionale
della scienza: evitare ogni forma di pregiudizio, procedere con ragionamenti chiari, stare ai fatti (ciò che
letteralmente è detto nel testo) e cercarne le cause reali e non immaginarie, condurre un analisi filologica della Bibbia.\footnote{E' stato il primo a compiere questo tipo di ricerca sul testo biblico in modo tanto approfondito.}

Il primo capito del Ttp discute della profezia biblica o rivelazione: questa è la conoscenza certa rivelata da Dio agli uomini. Equivale alla conoscenza naturale, ovvero le cose conosciute con la ragione dipendono dall'idea insita in noi di Dio e della natura di cui siamo parte, come spiegato per tutta l'Etica:

\begin{quotation}
	\small La conoscenza naturale si può
	chiamare divina allo stesso titolo di qualunque altra, perché essa ci viene come dettata dalla natura
	di Dio, in quanto ne siamo partecipi.\footnote{Ttp, prefazione, a cura di Alessandro Dini, Bompiani, 2019.}
\end{quotation}

Nella rivelazione biblica non è contenuto nulla di straordinario, bensì è un fenomeno naturale; il profeta è un "inviato e interprete" della natura quanto lo è lo scienziato o il filosofo.\\
Sbagliano: 
\begin{itemize}
	\item teologi,
	rabbini, preti e clero, che si attribuiscono capacità interpretative straordinarie relative a verità "trascendentali", che i profeti avrebbero rivelato oltre l'umana condizione ordinaria;
	\item il popolo superstizioso, che trascura e disprezza la conoscenza naturale perché "è comune a tutti gli uomini, in
	quanto poggia su basi che sono comuni a tutti"; molti uomini non intendono il carattere divino della
	conoscenza naturale, disprezzano i "doni naturali"
	della mente, preferendo invece l'insolito e il misterioso e assecondando la loro inclinazione 
	che li rendano assetati del miracoloso e dell'eccezionale, dai quali sperano aiuti soprannaturali ai suoi
	desideri e ai suoi timori;
	\item gli ebrei, che, nella loro presunzione e
	arroganza, si ritenevano superiori a tutti gli altri popoli e depositari dell'unica e vera rivelazione
	divina, disprezzando quella conoscenza naturale che è comune a tutti.
\end{itemize}

A questi errori va opposta la comprensione di Dio e della natura che possiamo cogliere tramite la conoscenza di secondo e terzo genere, ovvero capire che la mente dell'uomo è parte di quella divina afferrando i rapporti di natura e come si compone la realtà.

C'è però una differenza tra la conoscenza naturale comune e la rivelazione profetica. La prima non
deriva da Dio "a parole", non ha bisogno di profeti: tutti
possono apprenderla, senza bisogno di ricorrere alla fede. Dio, insomma, si è rivelato a parole ai profeti; ma si è anche rivelato, come diceva
Galileo, nel gran libro della natura e qui la ragione umana lo ritrova in modo assai più conforme
al suo proprio pensiero. Gli
uditori del profeta non possono riscoprire in se stessi la divina rivelazione (come accade nella
filosofia o nella scienza naturale), ma devono appoggiarsi "sulla testimonianza e sull'autorità del
profeta". Si tratta dunque di capire, servendosi della testimonianza della Scrittura e di essa sola, in
che consista quella forma di rivelazione divina che è la profezia biblica.

Dai testi sacri si ricava che Dio si rivelò ai profeti con parole o con figure, o con entrambe. Alla luce della lettura testuale della Scrittura
Spinoza sostiene che solo Mosè intese una "vera voce"\footnote{Ricordo che Mosè è l'autore del Pentateuco, la Torah, la parte "più importante" della Bibbia.}; per gli altri profeti  si trattò invece di immaginazioni, di sogni a occhi chiusi o aperti. Lo dice espressamente
la Scrittura: "Se qualcuno di voi sarà profeta di Dio, mi rivelerò a lui in sogno". 

Che la rivelazione mosaica inizi dalla voce è un fatto che non va sottovalutato o assunto come
qualcosa di ovvio. Nessun popolo ebbe forse così chiaro, come gli ebrei, il ruolo
fondamentale della voce nella costituzione dell'autocoscienza umana: da ciò la formazione dell'idea di una legge morale universale e la spiritualizzazione
del divino, non più disperso tra i corpi dei tori, dei leoni, e degli altri feticci
superstiziosamente adorati.

Da una rivelazione vocale seguì l'idea di una scrittura come
strumento di trasmissione della voce, a lei totalmente subordinato. La religione degli ebrei è una
religione della scrittura e del libro, ma in quanto legata alla voce divina; questa è una conseguenza di grande portata, responsabile
della coesione del popolo ebraico attraverso i secoli.

Sempre stando al testo della Scrittura, solo Cristo pervenne a
comunicare con Dio, non tramite la voce, ma in virtù della "sola mente", il che
testimonia della sua assoluta e unica eccellenza su tutti gli altri profeti.

Per tornare ai profeti del Vecchio Testamento, essi ebbero vivissima l'immaginazione:

\begin{quotation}
	\small Possiamo dunque ormai affermare senza riserve che i profeti non percepirono la rivelazione di Dio
	se non con l'aiuto dell'immaginazione, e cioè con parole e immagini, vere o immaginarie. E poiché
	non si trovano nella Scrittura altri mezzi all'infuori di questi, nessun altro, come abbiamo già
	dimostrato, ci è lecito escogitarne. Per quali leggi naturali, poi, ciò sia accaduto, confesso di
	ignorarlo. Potrei dire bensì, come altri dicono, che ciò è un effetto della potenza di Dio; ma mi
	parrebbe di parlare a vanvera. Sarebbe infatti come se io volessi spiegare con un termine
	trascendentale la forma di una cosa singolare. Tutto, in verità, è prodotto dalla potenza di Dio: anzi,
	poiché la potenza della Natura non è altro se non la stessa potenza di Dio, è certo che noi in tanto non intendiamo la potenza divina, in quanto ignoriamo le cause naturali; onde è stolto ricorrere
	alla potenza di Dio quando ignoriamo la causa naturale di una cosa, ossia la potenza divina stessa.
	D'altra parte, non è necessario che noi conosciamo la causa della conoscenza profetica, perché,
	come ho già avvertito, il nostro proposito è soltanto di esaminare i documenti della Scrittura, per
	trarne, come dai dati naturali, le nostre conclusioni. Le cause dei documenti non rientrano
	nell'ambito del nostro interesse. \footnote{Ttp, capitolo 1, a cura di Alessandro Dini, Bompiani, 2019.}
\end{quotation}

Nel secondo capitolo Spinoza dichiara apertamente lo scopo finale della sua opera: "la separazione della filosofia dalla teologia". Ciò comporta l'eliminazione di tutte le sovrastrutture filosofico-teologiche sovrapposte alla Bibbia, la riduzione della
Scrittura a un significato meramente morale, ad un insieme di precetti da osservare per una vita felice.

Ogni spiegazione della realtà deve risalire alle cause naturali, altrimenti
è solo superstizione: ne deriva che le profezie della Bibbia non
sono spiegazioni (non hanno nulla di scientifico), ma sono solo immaginazioni dei profeti. Queste
stesse immaginazioni, per essere spiegate, esigerebbero una conoscenza naturale della loro causa,
che però non abbiamo. Resta comunque stabilito che profezia e conoscenza naturale, pur essendo
entrambe "rivelazioni" risalenti a Dio, non stanno sullo
stesso piano. Afferma Spinoza che l'immaginazione profetica percepisce
molte cose che eccedono i limiti dell'intelletto. Infatti: 

\begin{quotation}
	\small dalle parole e dalle immagini si possono
	ricavare per composizione assai più idee che dai soli principi e dalle nozioni sulle quali si basa tutta
	la nostra conoscenza naturale.\footnote{Ttp, capitolo 2, a cura di Alessandro Dini, Bompiani, 2019.}
\end{quotation}

. Queste idee "immaginate" trovano poi la loro espressione
conforme in "parabole" e in "enigmi", cioè in tentativi di travestire materialmente le verità spirituali
rivelate. Sicché la profezia procede in direzione opposta alla scienza: questa punta alla
certezza chiara e distinta e alla comprensione della potenza di Dio; quella si fascia di misteri, di allusioni vaghe, di testimonianze e di discorsi che non potranno
mai essere ricondotti a una effettiva conoscenza del divino. Da ciò l'ovvia questione: se
la profezia è immaginazione,
da dove trassero i profeti la certezza del loro profetare e la verità divina
delle loro profezie?



I profeti non sono uomini dotati di una mente più perfetta
(Salomone, il più sapiente degli ebrei, non era profeta), ma sono uomini in possesso di grande immaginazione, che non implica di per sé alcuna certezza, come invece accade con le idee chiare e
distinte che ci fanno conoscere la natura.

\begin{quotation}
	\small \textbf{Tutta la certezza dei profeti era dunque fondata su questi tre motivi: }
	\begin{enumerate}
		\item che essi
		immaginavano le cose rivelate con vivacità pari a quella con la quale noi sogliamo essere affetti
		dagli oggetti allo stato di veglia;
		\item sul segno;
		\item \textbf{infine e soprattutto, che avevano l'animo inclinato
		soltanto all'equità e al bene.}
	\end{enumerate}
	
	E benché la Scrittura non faccia sempre menzione del segno, è da
	ritenere tuttavia che i profeti lo abbiano avuto sempre.\footnote{Ttp, capitolo 2, a cura di Alessandro Dini, Bompiani, 2019.}
\end{quotation}

Per i profeti la certezza era raggiunta
 grazie all'intervento di qualche segno che confermasse il contenuto e l'origine della loro
immaginazione. I segni e il contenuto stesso della
immaginazione profetica, per poter essere percepiti e compresi, dovevano essere "adeguati alle
opinioni e alle capacità del profeta stesso". Ecco perché i segni e i contenuti variano da profezia a
profezia, 

\begin{quotation}
	\small a seconda della costituzione fisica, dell'immaginazione e delle opinioni che (il profeta)
	aveva in precedenza professato. A un profeta allegro sono date da profetare vittorie, trionfi e
	letizie; a un profeta triste, sconfitte, sciagure e desolazioni.\footnote{Ttp, capitolo 2, a cura di Alessandro Dini, Bompiani, 2019.}
\end{quotation}

Corrispettivamente ogni profeta "vide Dio come era solito immaginarselo", cioè in base ai suoi
pregiudizi. Spinoza afferma che, facendo salva l'intenzione morale che guidava la profezia, il rivestimento immaginifico che essa assumeva nelle parole del profeta era solo un
fatto contingente e trascurabile, un mezzo che Dio liberamente usava per rivelarsi. Da cui la
conclusione generale: 

\begin{quotation}
	\small La profezia in nessun caso ha accresciuto la dottrina dei profeti, ma li ha
	lasciati sempre nelle loro opinioni preconcette, sicché noi, nelle cose di carattere meramente
	speculativo, non siamo affatto tenuti a prestarle fede.\footnote{Ttp, capitolo 2, a cura di Alessandro Dini, Bompiani, 2019.}
\end{quotation}

Dopo numerosi esempi presi dal testo biblico, Spinoza conclude:

\begin{quotation}
	\small Dalle cose dette risulta dunque abbondantemente ciò che ci eravamo proposti di
	dimostrare, e cioè che Dio ha adattato le sue rivelazioni alla comprensione e alle opinioni dei
	profeti; e che \textbf{i profeti poterono ignorare, e ignorarono di fatto, le cose concernenti la sola
	speculazione e non attinenti alla carità e alla pratica della vita}, e che professarono opinioni
	contrastanti. È dunque del tutto fuori luogo il voler attingere da essi una conoscenza delle cose
	naturali e spirituali. Concludiamo quindi che noi non siamo tenuti a credere ai profeti, se non in ciò
	che riguarda il fine e la sostanza della rivelazione. \footnote{Ttp, capitolo 2, a cura di Alessandro Dini, Bompiani, 2019.}
\end{quotation}

Dunque la separazione tra filosofia e teologia è compiuta, e quel che c'è di buono nelle sacre scritture sono solamente le regole morali presenti.

Fatta questa scissione i capitoli successivi chiariscono alcune questioni:

\begin{itemize}
	\item Il dono della profezia non è esclusivo del popolo ebreo, che non è il popolo eletto.
	\item Gli ebrei  si distinsero soltanto per aver ben amministrato riguardo la sicurezza e aver scansato i pericoli.
	\item  La legge divina naturale (che acquistiamo con il secondo e terzo grado di conoscenza) è universale e alla portata di chiunque si applichi, non richiede fede ma solo sforzo della ragione.
	\item "Colui che fa il bene in base alla vera conoscenza e all'amore del bene, agisce liberamente e con fermezza d'animo, mentre chi lo fa per timore del male (perché sottoposto ad una legge morale esterna che non comprende), costui agisce sotto la costrizione del male e da schiavo, e vive sotto il potere di un altro."\footnote{Ttp, capitolo 4, a cura di Alessandro Dini, Bompiani, 2019.}
	\item Le cerimonie e i rituali non servano alla beatitudine. Essi sono creati affinché gli uomini non agissero per scelta propria, ma facessero ogni cosa per comandamento dei potenti. Alcune volte questa situazione non deve essere interpretata negativamente, come il caso di Mosè che mirava a far agire gli ebrei ai suoi comandamenti, altrimenti mai si sarebbero destreggiati nel mondo, appena usciti dallo stato di schiavitù cui sottostavano in Egitto.
	\item \textbf{La legge è stata legata alla religione per indurre i sudditi ad un'obbedienza spontanea; cercare il rispetto delle regole con il timore è meno efficiente e produttivo.}
	\item Gli insegnamenti della Scrittura sono comunicati con il primo genere di conoscenza, raccontando storie ed esperienze: questo è il modo più semplice di comunicare con il "volgo", che viene mosso ad obbedienza e devozione.
	\item I miracoli non sono mai avvenuti; l'accadimento di un miracolo significherebbe che Dio operi contro se stesso e la sua natura necessaria, il che è assurdo. Per il popolo, l'esistenza di un Dio antropomorfo, che agisce a somiglianza del comportamento umano, è dimostrata dal fatto che la natura non conservi il proprio ordine: quindi nelle scritture i miracoli sono presenti per catturare l'attenzione del volgo, per persuadere alla fede, ma anche per la semplice incapacità dello scrittore di spiegare eventi naturali. Per apprendere l'esistenza di Dio non vi è niente di meglio dello studio dell'ordine naturale; in quest'ottica il miracolo contro natura fa dubitare di Dio.
\end{itemize}

Procedendo nel Ttp, al capitolo sette, Spinoza approfondisce il metodo interpretativo delle Scritture. 

\begin{quotation}
	\small Se gli uomini fossero sinceri nella testimonianza che essi a parole rendono della
	Scrittura, seguirebbero una ben diversa regola di vita: i loro animi non sarebbero
	agitati da tante discordie, non si combatterebbero gli uni gli altri con tanto odio
	e non sarebbero accesi da una così cieca e temeraria smania di interpretare i Sacri
	Testi e di scogitare nuovi dogmi nella religione; al contrario, essi non oserebbero
	accogliere come dottrina della Scrittura se non ciò che fosse in modo evidentissimo
	insegnato da essa.\footnote{Ttp, capitolo 7, a cura di Alessandro Dini, Bompiani, 2019.}
\end{quotation}

Gli unici strumenti ermeneutici sono quelli forniti dall'intelligenza e dal raziocinio applicati al testo, senza nessun "lume soprannaturale né un’autorità esterna":

\begin{quotation}
	\small Quindi \textbf{la conoscenza di
	tutto ciò, ossia di quasi tutto quanto è contenuto nella Scrittura, va ricavata
	esclusivamente dalla scrittura stessa}, come la conoscenza della natura dalla sola
	natura.\footnote{Ttp, capitolo 7, a cura di Alessandro Dini, Bompiani, 2019.}
\end{quotation}

Ciò significa dover superare problemi interpretativi dovuti alla conoscenza dell'ebraico antico e degli autori dei libri componenti la Bibbia, all'incerta autenticità di alcune sue parti. Spinoza affronta il problema cercando di trarre dalla storia delle Scritture la dottrina fondamentale proposta, che servirà poi a chiarire il senso dei singoli episodi e interpretarli correttamente. Le conclusioni che trae sono che nella Bibbia:

\begin{quotation}
	\small i veri insegnamenti della pietà sono espressi con parole usitatissime (usate molto frequentemente), tanto sono comuni, semplici e facili a intendersi; e poiché la vera salvezza e beatitudine consistono nella vera tranquillità dell'animo, e noi troviamo davvero quiete soltanto in quelle cose che intendiamo in maniera chiarissima, ne segue nella maniera più evidente che noi di sicuro possiamo giungere a comprendere il pensiero della Scrittura riguado alle cose necessarie alla salvezza e alla beatitudine.\footnote{Ttp, capitolo 7, a cura di Alessandro Dini, Bompiani, 2019.}
\end{quotation}

Il filo conduttore "chiaro e distinto" di tutti i testi sacri sono i valori morali di pietà, fratellanza, aiuto reciproco e amore verso il prossimo. Questo ci dice Spinoza:

\begin{quotation}
	\small Dalla stessa Scrittura, senza alcuna difficoltà e ambiguità,noi percepiamo che \textbf{l'essenza dell'insegnamento è amare Dio sopra ogni cosa e il prossimo come se stessi}.\footnote{Ttp, capitolo 12, a cura di Alessandro Dini, Bompiani, 2019.}
 \end{quotation}

I capitoli dall'ottavo all'undicesimo del Ttp esaminano tutte le Scritture, dal Vecchio al Nuovo Testamento, per dimostrare le tesi spinoziane: l'analisi è fatta con molta accuratezza e precisione, evidenziando lacune ed errori, cambiamenti del linguaggio, incoerenze.

Altro aspetto che emerge, è l'inutilità di interpretazioni astruse che vedono profondissimi misteri nelle Scritture: il Talmud e la Cabala per il mondo ebraico, e in generale tutte le disquisizioni filosofiche fatte sulle Scritture, sono frutto della superstizione.
Ai primi ebrei la religione fu data come legge scritta, "perché a quel tempo erano come dei bambini", che capiscono le cose per mezzo delle immagini. La Scrittura fu fatta non solo per i dotti, ma per tutti gli uomini di qualunque età e genere, per cui le interpretazioni metaforiche dei filosofi non hanno senso, così come  prendere alla lettera quanto sfugge alle loro capacità; non era intenzione dei profeti fare filosofia.

\begin{quotation}
	\small Niente fuori dalla mente è in assoluto sacro o profano o impuro, ma soltanto in rapporto ad essa.\footnote{Ttp, capitolo 12, a cura di Alessandro Dini, Bompiani, 2019.}
\end{quotation}

In questa citazione, Spinoza comunica che la Scrittura è sacra finché muove gli uomini alla devozione verso Dio; se questa viene a mancare, se gli uomini non perseguono pietà e amore,  la Bibbia non è altro che carta e inchiostro.
Quindi la fede consiste nel credere in Dio, richiede ubbidienza ai suoi insegnamenti, che ribadiamo essere carità e amore verso gli altri. Da questo derivano due cose: la prima che la fede da salvezza non in quanto tale, ma in rapporto all'ubbidienza, ovvero la sola fede senza opere è morta; la seconda che colui che è veramente ubbidiente alle Scritture (e quindi pratica opere), necessariamente ha la fede vera e salvifica. Dal vangelo di Giovanni (4, 7-8) Spinoza cita: "Chiunque ama il prossimo è nato da Dio e conosce Dio, e chi non lo ama non lo conosce; infatti Dio è amore".

Solo dalle opere possiamo giudicare la fede di un uomo: se sono buone (in rapporto agli insegnamenti delle Scritture), allora egli è credente, sebbene dissenta per i dogmi dagli altri credenti.

\begin{quotation}
	\small Posta l'ubbidienza, necessariamente è posta la fede, e la fede senza opere è morta.\footnote{Ttp, capitolo 14, a cura di Alessandro Dini, Bompiani, 2019.}
\end{quotation}

Essendo l'indole umana assai varia tanto da avere giudizi spesso contrastanti su tutto, segue che alla fede universale non devono appartenere dogmi, che possono variare secondo le disposizioni naturali di ciascuno, bensì la fede va giudicata solo in rapporto all'ubbidienza, attraverso i fatti. Qui è chiaro l'intento in Spinoza di definire un credo universale per evitare controversie e favorire la pace nelle comunità.

La filosofia, cioè la conoscenza intellettuale di Dio tramite le idee adeguate, e la fede, portano entrambe alla salvezza: la prima grazie alla ragione, la seconda tramite l'ubbidienza alle Scritture. Il fine perseguito è lo stesso, completamente diverse sono le strade che vi conducono.

Visto che il contenuto della rivelazione non può essere fondato razionalmente, Spinoza si domanda quali siano le ragioni per credere  o perché le Scritture siano credibili: l'autorità dei testi sacri deriva dai profeti che hanno testimoniato con la loro vita la validità del loro insegnamento, e dal fatto che la "parola di Dio nei profeti concorda del tutto con la stessa parola di Dio che parla in noi".
Come si era detto all'inizio di questa parte, il principio di autorità dei profeti si basava su immaginazione, sui segni e "soprattutto nell'animo incline alla giustizia e al bene", non su altre ragioni.

\section[Parte politica]{Parte politica del Ttp}

Una volta dimostrato che l’individuo gode di libertà nell’ambito dell’interpretazione religiosa e che la libertà di pensiero e la libertà
di filosofare non devono dipendere dalla religione e dalla teologia, il passo successivo sarà quello di studiare fino a che punto possano svilupparsi, in ambito
sociale, le libertà di pensiero e d’espressione; o, detto con parole dello stesso
Spinoza, "è ora tempo di cercare fin dove si estenda questa libertà di pensare
e di dire quello che si pensa in uno Stato ben ordinato". Prima di continuare, è tuttavia necessario sapere qual è, secondo Spinoza, "il migliore Stato", e
dunque un’indagine intorno ai fondamenti dello stesso. Si tratta di domandarsi
quale modello di Stato sia "il più naturale e quello che più si avvicina alla libertà che la natura concede a ciascun individuo".\footnote{Entrambi i virgolettati provengono da: Ttp, capitolo 16, a cura di Alessandro Dini, Bompiani, 2019.}












\newpage