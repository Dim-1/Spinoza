\chapter{Spinoza contemporaneo}

Oggi si tende a generare confusione tra piano della percezione e piano della realtà. Siamo convinti che per cambiare i fatti, basti variare le parole: così la friggitrice ad aria "frigge" senza olio; il ministero della difesa non fa la guerra; in Ucraina la Russia sta conducendo un'operazione speciale.

Marketing, politica, tutte le attività che si basano sul convincere il prossimo, sfruttano questi principi relativistici. Ancor più oggi, Spinoza ci può essere d'insegnamento, con il suo costante stimolo a conoscere, per spazzare via superstizioni, fallacie, indottrinamenti o semplicemente per non farsi raggirare.

Ogni volta che squarciamo il velo dell'ignoranza, ci appropriamo della conoscenza, e della libertà di sapere se una cosa sia per noi buona o cattiva, utile o dannosa. Senza questa consapevolezza, ogni cosa che ignoriamo, ci viene imposta da regole morali esterne alla nostra mente, che non capiamo e che ci sottomettano.








\newpage