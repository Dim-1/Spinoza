\chapter{Spinoza contemporaneo}

Oggi si tende a generare confusione tra piano della percezione e piano della realtà. Siamo convinti che per cambiare i fatti, basti variare le parole: così la friggitrice ad aria "frigge" senza olio; il ministero della difesa non fa la guerra; in Ucraina la Russia sta conducendo un'operazione speciale.

Marketing, politica, tutte le attività che si basano sul convincere il prossimo, sfruttano questi principi relativistici. Ancor più oggi, Spinoza ci può essere d'insegnamento, con il suo costante stimolo a conoscere, per spazzare via superstizioni, fallacie, indottrinamenti o semplicemente per non farsi raggirare.

Ogni volta che squarciamo il velo dell'ignoranza, ci appropriamo della conoscenza, e della libertà di sapere se una cosa sia per noi buona o cattiva, utile o dannosa. Senza questa consapevolezza, ogni cosa che ignoriamo, ci viene imposta da regole morali esterne alla nostra mente, che non capiamo e che ci sottomettano.

Sorge adesso la domanda: è ancora rilevante affermare la libertà della espressione del libero pensiero? La licenza di
esprimere tutto ciò che si vuole con la potenza dei mezzi tecnici della informazione e della
cultura di massa non produce affatto libertà, pietà e pace. L'alfabetizzazione universale e
l'immaginario culturale supportato dai mass media hanno
trasformato radicalmente la situazione rispetto ai tempi di Spinoza, quando la popolazione era in
prevalenza analfabeta e il libro era appannaggio di pochi. La libertà di espressione, un tempo nobile
principio, si traduce di fatto nella schiavizzazione coatta dei
più sprovveduti e dei più deboli, cioè in generale dei più giovani.

La difesa del valore
“democratico” della libera espressione fatta da Spinoza, oggi potrebbe suonare come una irrisione e come una truffa grottesca. Egli  difendeva la "libertà di filosofare": le
odierne condizioni di vita e di cultura rendono di fatto obsoleto, sconosciuto e introvabile
ogni autentico filosofare; proprio la libera e indiscriminata espressione
rende desueta e alla lunga impossibile la pratica filosofica.

Ovviamente la censura non è una soluzione, i tradizionalisti che la evocano sono soltanto uomini intolleranti e violenti. Spinoza, con "libertà di filosofare", invoca un'\textbf{emozione razionale} che ha supportato gran parte della storia
dell'Occidente, una passione per il ragionamento che ha bisogno sia di libertà (che oggi abbiamo), che di etica del pensare (che oggi manca): occorre offrire alle persone i mezzi culturali per progredire, lo stimolo alla riflessione e al ragionamento, altrimenti la libertà di pensiero si trasforma in libertà di scegliere chi ci controlla.

Un aspetto di Spinoza apparentemente non al passo con i tempi è il suo estremo determinismo. Provo ad esaminare la questione ricorrendo alla mia personale esperienza scientifica: al tempo dei miei studi universitari in chimica, durante il corso di chimica biologica, rimasi colpito dall'estrema specificità di ogni singolo enzima. Gli enzimi sono delle proteine costruiti dalle cellule viventi, che vanno a rompere o formare legami chimici all'interno delle gigantesche macromolecole biologiche. La specificità degli enzimi è tale che esiste un enzima diverso per ogni tipo di legame biologico: ad esempio, per meglio comprendere, esiste un enzima specifico per rompere il legame peptidico per ogni coppia possibile di amminoacidi (che sono 20, quindi abbiamo 400 combinazioni possibili). Visto che il legame peptidico è sempre uguale, e che i vari amminoacidi si differenziano per le catene laterali (che non prendono parte al suddetto legame), rimasi stupito da questa estrema specificità e compresi che in natura niente avviene per caso. Considerando quanta precisione  governa i processi della vita, il determinismo  non appare più una concezione sbagliata, descrivendo gli accadimenti della realtà, fisica o morale, reciprocamente connessi in modo necessario e invariabile.

La questione diventa allora se tutta l'esistenza determinata è determinabile: sempre prendendo in esempio le cellule viventi, queste sono composte da talmente tante e numerose molecole (al loro volta composte da numeri enormi di atomi), da rendere estremamente dispendioso e difficoltoso progredire nella catena causa-effetto in entrambe le direzioni, perché essa è si finita, ma ha un valore talmente elevato da rasentare l'infinito. Perciò acquistano grande valore le scienze umane, che ci permettano di conoscere nella pratica ciò che potrebbe essere compreso ma che non può essere determinato con precisione per l'eccessiva difficoltà del calcolo: da un punto di vista spinoziano, le scienze umane studiano gli effetti e risalgono solo una parte delle quasi infinite concatenazioni causa-effetto. La dottrina del \textit{conatus} è il tentativo riuscito di spiegare il progredire necessario della natura senza partire dalla singola molecola biologica.

Spinoza ci fornisce dei mezzi esistenziali per dare un significato alla vita, creando \textbf{una religione della ragione}.\footnote{Spinoza è stato il primo nella storia del pensiero umano a fondare una religione della ragione, forse prima di lui gli stoici avevano una concezione simile della vita, ma basata su un sistema filosofico non ben fondato quanto quello spinoziano.} Da Blaise Pascal (vissuto quasi contemporaneo a Spinoza, 1623-1662) in poi quasi tutte le risposte alle domande esistenziali hanno trovato nel Dio cristiano le motivazioni e le soluzioni; Spinoza cerca nel mondo naturale e nelle sue leggi le spiegazioni che Pascal trovò in un Dio volitivo e trascendente. Siamo al mondo in quanto manifestazioni della natura, dell'universo, momenti passeggeri destinati a trasformarsi e magari ritornare, vista la permanenza eterna delle nostre essenze, da intendersi come sequenza di passaggi necessari che hanno portato alla nostra esistenza\footnote{Friedrich Nietzsche e la teoria dell'eterno ritorno della vita forse devono qualcosa a Spinoza.}. Tutto ciò va compreso, interiorizzato e accettato, per vivere con gioia il nostro tempo: dobbiamo ragionare su tutto quello che ci ha portato ad essere qui ora e adesso. Raggiunta la comprensione arriverà la felicità, figlia di quella piccola gioia che proviamo ogni volta che risolviamo anche un piccolissimo problema con l'uso del nostro cervello. L'incomprensione delle leggi naturali e della vita ci conduce alla sconfitta, perché saremo sempre schiacciati da delle potenze che appaino insormontabili, vista la non conoscenza del mondo; l'ignoranza sicuramente ci vede sconfitti al momento della morte o anche prima del suo sopraggiungere, se viviamo nella paura che il trapasso genera. 

Questa religione che conduce  all'accettazione di ciò che si è, quindi genera tolleranza e consenso verso tutto il mondo, altri umani compresi, come Spinoza ha ampiamente dimostrato.

Riguardo la connessione mente-corpo che per primo Spinoza ha teorizzato, voglio far notare l'incredibile innovazione che non tutti hanno ben colto: della prova scientifica del rapporto tra emozioni corporali ed sensazioni della mente ne ho parlato nel capitolo dedicato all'Etica; mi preme evidenziare l'importanza che Spinoza assegna alle affezioni provenienti dall'esterno per renderci più consapevoli di noi stessi e quindi aumentare il proprio grado di potenza. Queste interazioni tra il mondo e il nostro corpo sono ovviamente generate dal movimento, per cui riporto questa citazione dello psicologo Théodule-Armand Ribot: 

\begin{quotation}
	\small Senza elementi motori la percezione è impossibile. Se l’occhio è tenuto fisso sopra un dato oggetto, la percezione dopo qualche tempo diviene confusa e infine sparisce del tutto. Se si posa sul tavolo, senza premere, la punta delle dita, dopo pochi minuti non si avvertirà più il contatto. Ma basta un movimento anche minimo dell’occhio o delle dita perché la percezione si risvegli. Può esservi coscienza soltanto dove c’è cambiamento e può esservi cambiamento solo dove c’è movimento. Sarebbe facile diffondersi a lungo sopra un tale argomento; poiché, sebbene i fatti siano evidentissimi e di comune esperienza, la psicologia nondimeno ha così trascurato l’importanza dei movimenti da giungere finanche a dimenticare che essi sono la condizione fondamentale della cognizione, essendo lo strumento della legge fondamentale della coscienza, che è la relatività, il mutamento. Alla luce di tutto ciò che è stato fin qui detto, si può affermare incondizionatamente che dove non c’è movimento non c’è percezione.\footnote{Théodule-Armand Ribot, Psicologia dell'attenzione.}
\end{quotation}

In ultimo mi preme riflettere sul rapporto tra etica e morale: come sarebbe oggi vivere comportandosi nella vita quotidiana secondo ragione (perciò con etica) piuttosto che secondo morale?\\
La regole morali sono imposte da leggi e tradizioni che spesso pensiamo essere antiche, sempre esistite, eterne e quindi non modificabili ne violabili apertamente. Spesso infrangiamo i precetti morali della nostra comunità, ma non desideriamo farlo sapere al mondo, il che fa di noi dei moralisti: la morale intorno alla famiglia è un caso emblematico delle contraddizioni del nostro tempo.
Affrontiamo eticamente questo contrasto e chiediamoci quale è il senso della famiglia: perché dobbiamo amare poche persone quando  potrei "compormi" con molte altre, perché preferire poco, piccolo e privato piuttosto che grande e comune?



























\newpage