\chapter{Spinoza contemporaneo}

Oggi si tende a generare confusione tra piano della percezione e piano della realtà. Siamo convinti che per cambiare i fatti, basti variare le parole: così la friggitrice ad aria "frigge" senza olio; il ministero della difesa non fa la guerra; in Ucraina la Russia sta conducendo un'operazione speciale.

Marketing, politica, tutte le attività che si basano sul convincere il prossimo, sfruttano questi principi relativistici. Ancor più oggi, Spinoza ci può essere d'insegnamento, con il suo costante stimolo a conoscere, per spazzare via superstizioni, fallacie, indottrinamenti o semplicemente per non farsi raggirare.

Ogni volta che squarciamo il velo dell'ignoranza, ci appropriamo della conoscenza, e della libertà di sapere se una cosa sia per noi buona o cattiva, utile o dannosa. Senza questa consapevolezza, ogni cosa che ignoriamo, ci viene imposta da regole morali esterne alla nostra mente, che non capiamo e che ci sottomettano.

Sorge adesso la domanda: è ancora rilevante affermare la libertà della espressione del libero pensiero? La licenza di
esprimere tutto ciò che si vuole con la potenza dei mezzi tecnici della informazione e della
cultura di massa non produce affatto libertà, pietà e pace. L'alfabetizzazione universale e
l'immaginario culturale supportato dai mass media hanno
trasformato radicalmente la situazione rispetto ai tempi di Spinoza, quando la popolazione era in
prevalenza analfabeta e il libro era appannaggio di pochi. La libertà di espressione, un tempo nobile
principio, si traduce di fatto nella schiavizzazione coatta dei
più sprovveduti e dei più deboli, cioè in generale dei più giovani.

La difesa del valore
“democratico” della libera espressione fatta da Spinoza, oggi potrebbe suonare come una irrisione e come una truffa grottesca. Egli  difendeva la "libertà di filosofare": le
odierne condizioni di vita e di cultura rendono di fatto obsoleto, sconosciuto e introvabile
ogni autentico filosofare; proprio la libera e indiscriminata espressione
rende desueta e alla lunga impossibile la pratica filosofica.

Ovviamente la censura non è una soluzione, i tradizionalisti che la evocano sono soltanto uomini intolleranti e violenti. Spinoza, con "libertà di filosofare", invoca un'\textbf{emozione razionale} che ha supportato gran parte della storia
dell'Occidente, una passione per il ragionamento che ha bisogno sia di libertà (che oggi abbiamo), che di etica del pensare (che oggi manca): occorre offrire alle persone i mezzi culturali per progredire, lo stimolo alla riflessione e al ragionamento, altrimenti la libertà di pensiero si trasforma in libertà di scegliere chi ci controlla.

Un aspetto di Spinoza apparentemente non al passo con i tempi è il suo estremo determinismo. Provo ad esaminare la questione ricorrendo alla mia personale esperienza scientifica: al tempo dei miei studi universitari in chimica, durante il corso di chimica biologica, rimasi colpito dall'estrema specificità di ogni singolo enzima. Gli enzimi sono delle proteine costituiti dalle cellule viventi, che vanno a rompere o formare legami chimici all'interno delle gigantesche macromolecole biologiche. La specificità degli enzimi è tale che esiste un enzima diverso per ogni tipo di legame biologico: ad esempio, per meglio comprendere, esiste un enzima specifico per rompere il legame peptidico per ogni coppia possibile di amminoacidi (che sono 20, quindi abbiamo 400 combinazioni possibili). Visto che il legame peptidico è sempre uguale, e che i vari amminoacidi si differenziano per le catene laterali (che non prendono parte al suddetto legame), rimasi stupito da questa estrema specificità e compresi che in natura non avviene niente per caso. Considerando quanta precisione  governa i processi della vita, il determinismo assume molta più credibilità.

La questione diventa allora se tutta l'esistenza determinata è determinabile: sempre prendendo in esempio le cellule viventi, queste sono composte da talmente tante e numerose molecole (al loro volta composte da numeri enormi di atomi), da rendere estremamente dispendioso e difficoltoso progredire nella catena causa-effetto in entrambe le direzioni, perché essa è si finita, ma ha un valore talmente elevato da rasentare l'infinito. Perciò acquistano valore le scienze umane, che ci permettano di conoscere nella pratica ciò che è esatto ma che non può essere determinato con precisione per l'eccessiva difficoltà del calcolo: da un punto di vista spinoziano, le scienze umane studiano gli effetti o risalgono solo una parte delle quasi infinite concatenazioni causa-effetto. La dottrina del \textit{conatus} è il tentativo riuscito di spiegare il progredire necessario della natura senza partire dalla singola molecola biologica.














\newpage