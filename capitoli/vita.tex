\chapter*{Vita}
Barruch Spinoza nacque ad Amsterdam il 24 novembre 1632, da una famiglia di origine Portoghese (ma proveniente dalla Spagna), emigrata nelle allora Provincie Unite, oggi Olanda, per scappare dalle persecuzioni della corona Spagnola verso i marrani, gli ebrei convertiti al cristianesimo.

Oltre a lavorare nell'attività commerciale della sua famiglia, nel 1654 iniziò a prendere lezioni di latino da Franciscus Van den Enden. Conobbe così i classici della filosofia, in particolare di quella scritta in latino: Seneca, Orazio, Cesare, Virgilio, Tacito, Epitteto, Livio, Plinio, Ovidio, Cicerone, Marziale, Petrarca, Petronio, Sallustio.

A causa di idee considerate eretiche dalla sinagoga locale, il 27 luglio 1656 Spinoza venne scomunicato:
\begin{quotation}
	\small I Signori del Mahamad rendono noto che, venuti a conoscenza già da tempo delle cattive opinioni e del comportamento di Baruch Spinoza, hanno tentato in diversi modi e anche con promesse di distoglierlo dalla cattiva strada. Non essendovi riusciti e ricevendo, al contrario, ogni giorno informazioni sempre maggiori sulle orribili eresie che egli sosteneva e insegnava e sulle azioni mostruose che commetteva – cose delle quali esistono testimoni degni di fede che hanno deposto e testimoniato anche in presenza del suddetto Spinoza – questi è stato riconosciuto colpevole. Avendo esaminato tutto ciò in presenza dei Signori Rabbini, i Signori del Mahamad hanno deciso, con l'accordo dei Rabbini, che il nominato Spinoza sarebbe stato bandito e separato dalla Nazione d'Israele in conseguenza della scomunica che pronunciamo adesso nei termini che seguono:
	
	Con l'aiuto del giudizio dei santi e degli angeli, con il consenso di tutta la santa comunità e al cospetto di tutti i nostri Sacri Testi e dei 613 comandamenti che vi sono contenuti, escludiamo, espelliamo, malediciamo ed esecriamo Baruch Spinoza. Pronunciamo questo herem nel modo in cui Giosuè lo pronunciò contro Gerico. Lo malediciamo nel modo in cui Eliseo ha maledetto i ragazzi e con tutte le maledizioni che si trovano nella Legge. Che sia maledetto di giorno e di notte, mentre dorme e quando veglia, quando entra e quando esce. Che l'Eterno non lo perdoni mai. Che l'Eterno accenda contro quest'uomo la sua collera e riversi su di lui tutti i mali menzionati nel libro della Legge; che il suo nome sia per sempre cancellato da questo mondo e che piaccia a Dio di separarlo da tutte le tribù di Israele affliggendolo con tutte le maledizioni contenute nella Legge. E quanto a voi che restate devoti all'Eterno, vostro Dio, che Egli vi conservi in vita. Sappiate che non dovete avere con Spinoza alcun rapporto né scritto né orale. Che non gli sia reso alcun servizio e che nessuno si avvicini a lui più di quattro gomiti. Che nessuno dimori sotto il suo stesso tetto e che nessuno legga alcuno dei suoi scritti. \footnote{ Emilia Giancotti Boscherini, Baruch Spinoza 1632-1677,Dichiarazione rabbinica autentica datata 27 luglio 1656 e firmata da Rabbi Saul Levi Morteira ed altri, Roma, Editori Riuniti 1985, p. 13 e sgg.}
\end{quotation}

La scomunica lo costrinse a lasciare Amsterdam, per recarsi all'Aja, dopo poche tappe intermedie: dormì sempre in camere di alberghi, e per mantenersi faceva il tornitore di lenti, lavoro all'avanguardia nel '600. Non accettò mai grosse rendite che gli furono offerte, se non parzialmente, e rifiutò anche incarichi universitari importanti: tutto per mantenere la sua libertà intellettuale, aspetto molto importante in tutto il pensiero spinoziano.

All'età di 29 anni e dopo la drammatica espulsione dalla comunità ebraica, Spinoza pubblica i "Principi della filosofia di Cartesio", con l'appendice "Cogitata metaphysica", in cui interpreta la filosofia cartesiana. In questo momento già aveva un gruppo di amici e discepoli, con cui intratteneva scambi epistolari, fonte importante per ricostruite il pensiero di Spinoza.

"Ethica more geometrico demonstrata" fu iniziata a scrivere nel 1661, ma non verrà mai pubblicata con Barruch in vita, a causa del suo contenuto che sarebbe stato considerato empio da tutte le comunità religione. Infatti la pubblicazione anonima del "Trattato teologico-politico" (seconda e ultima opera pubblicata in vita) nel 1670 suscitò clamori sia tra gli ebrei che tra i cristiani di ogni confessione, a causa del contenuto che metteva in dubbio la verità delle Scritture: l'identificazione dell'autore in Spinoza fu rapida, e questo gli portò addosso accuse di essere ateo e blasfemo.

Nel 1672 vide molti amici di ideali repubblicani (fra questi i fratelli de Witt) morire durante l'ascesa del partito monarchico degli Orange.

Spinoza, affetto da congeniti disturbi respiratori, aggravati dalla polvere di vetro inalata nel taglio delle lenti, morì di tubercolosi il 21 febbraio 1677, all'età di 44 anni. Nello stesso anno i suoi amici pubblicarono postumi: "Ethica", "Trattato sull'emendazione dell'intelletto" (incompiuto, iniziato in età giovanile per poi lasciare il posto alla stesura del Ethica), il "Trattato politico" (incompiuto al capitolo della democrazia causa la morte dell'autore), l'Epistolario e una grammatica ebraica, il "Compendio di lingua ebraica".

In Spinoza si concretizza l’identità tra filosofia e vita, come in pochi altri pensatori prima di lui (Socrate, Diogene, Epicuro i primi che mi vengono in mente); e la ricerca filosofica ad altro non tende se non alla realizzazione della vita buona.