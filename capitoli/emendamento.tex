\chapter{Trattato sull'emendazione dell'intelletto}

\bigskip
\bigskip
\bigskip

Come il titolo lascia intuire, lo scopo di questo trattato è la correzione di errori e pregiudizi radicate nelle credenze umane, che impediscano di cogliere la verità e il bene.

Qui tratterò solamente il prologo di quest'opera, mentre il resto verrà tralasciato: perché di non facile comprensione; perché l'esposizione del "metodo" spinoziano qui esposto verrà ampiamente superato  e perfezionato nell'Etica, tanto che il trattato non è stato portato a termine dal suo autore.

\begin{quotation}
	\small Dopo che l'Esperienza mi ebbe insegnato che tutte le cose che accadono normalmente nella vita
	comune sono vane e futili; e quando ebbi visto che tutto ciò che temevo e che generava in me
	inquietudine non aveva niente di buono né di malvagio in sé, ma solo in quanto l'animo ne era
	agitato; decisi infine di indagare se si desse qualcosa che fosse il vero bene, che fosse attingibile di
	per sé, e da cui solo, abbandonati tutti gli altri, l'animo potesse essere affetto; e insomma se si desse
	qualcosa per mezzo del quale, una volta trovatolo e raggiuntolo, potessi godere in eterno di continua
	e perfetta felicità.\footnote{Trattato sull'emendazione dell'intelletto, traduzione e cura di Michele Lavazza.}
\end{quotation}

Le prime righe di questo scritto ricordano il Discorso sul Metodo di Cartesio, in cui, con stile autobiografico, l'autore lamenta l'insoddisfazione verso le norme d'insegnamento e di conoscenza tradizionali insegnate nelle scuole di inizio 1600.

Il giovane Spinoza si rende conto che la felicità non può risiedere negli "agi che derivano dagli
onori e dalle ricchezze": sono beni non duraturi, che potrebbero sparire in ogni momento, sempre che donino effettivamente la felicità. Ogni stile di vita che cerca un compresso senza sconvolgere le abitudini quotidiane è destinato a fallire, perché ricchezze, onori e piaceri occupano troppo la mente

\begin{quotation}
	\small tra le cose che si concretizzano nella vita,quelle che presso gli uomini vengono
	considerate alla stregua del sommo bene si riducono a queste tre: le ricchezze, gli onori e i piaceri
	sensibili. Da esse la mente è a tal punto assorbita che può a mala pena pensare a qualche altro bene. Infatti ciò che riguarda i piaceri lascia l'animo a tal punto sospeso, come se riposasse in un vero
	bene, che esso è del tutto incapace di pensare ad altro; ma, dopo la fruizione di tali piaceri, subentra
	una somma tristezza che, se non tiene altrettanto in sospeso la mente, tuttavia la turba e inebetisce.\\
	Anche perseguendo gli onori e le ricchezze 3 la mente è non poco distratta, e soprattutto
	quando essi, identificati con il sommo bene, non sono ricercati se non per sé; la mente invero è
	assorbita dagli onori ancora molto di più: si ritiene infatti sempre che essi siano buoni di per sé, ed
	essi sono considerati come fini ultimi verso cui tutto deve tendere. Inoltre a onori e ricchezze non è
	associata, come ai piaceri sensibili, una penitenza; bensì a colui che possiede la maggior quantità di
	entrambi spetta tanto maggiore felicità, e di conseguenza siamo sempre più incoraggiati a
	incrementarli entrambi: e se le nostre speranze risultano in qualche modo frustrate, ciò è causa di
	grande tristezza.\\
	La ricerca degli onori è insomma di grande intralcio poiché, per ottenerli, la vita deve necessariamente essere condotta secondo le abitudini dei più, evitando quello che evita il volgo,
	cercando quello che il volgo cerca.\footnote{Trattato sull'emendazione dell'intelletto, traduzione e cura di Michele Lavazza.}
\end{quotation}

La vera felicità deve derivare da beni non passeggeri, non "transeunti", ma da un bene "incerto, ma non per sua natura (cercavo infatti
un bene immutabile), bensì solamente quanto alla sua acquisizione". Infatti: 

\begin{quotation}
	\small l'amore nei confronti di qualcosa di eterno e infinito nutre l'animo di pura felicità e per mezzo di
	essa sola tutta la tristezza è dissipata; il che è da desiderarsi intensamente e da perseguire con tutte
	le forze.\footnote{Trattato sull'emendazione dell'intelletto, traduzione e cura di Michele Lavazza.}
\end{quotation}

Man mano che il vero bene si dispiega alla mente, ci rendiamo conto che:

\begin{quotation}
	\small l'acquisizione di denaro, o la sregolatezza, o la sete di gloria nuocciono solo
	fintantoché sono ricercate per se stesse e non in quanto mezzo per altri scopi; se infatti sono
	ricercate in vista di altro, allora sono perseguite con moderazione e non nuocciono per nulla, anzi
	molto favoriscono il raggiungimento dello scopo per cui sono ricercate, come mostreremo a tempo
	debito.\footnote{Trattato sull'emendazione dell'intelletto, traduzione e cura di Michele Lavazza.}
\end{quotation}

Dunque per Spinoza il sommo bene da desiderare per una vita appagata e felice è Dio, da non intendersi assolutamente con il Dio biblico o cristiano, antropomorfo e dotato di volizione; la sua natura coincide con la "conoscenza dell'unità tra la mente e la Natura nel suo complesso". Già in quest'opera si capisce la centralità dei rapporti di natura necessari che governano il mondo, la cui comprensione dona all'uomo quella consapevolezza fonte di felicità, in quanto gli permette di esprimere interamente la sua potenza.

Questa scoperta del sommo bene deve essere condivisa. Per Spinoza non esiste gioia personale, ma solo universale.

\begin{quotation}
	\small Qui sta dunque lo scopo a cui tendo, raggiungere cioè
	questa natura, e tentare di far sì che molti la raggiungano con me; cioè la mia felicità dipende anche
	dal mio adoperarmi affinché molti altri, come me, capiscano.\footnote{Trattato sull'emendazione dell'intelletto, traduzione e cura di Michele Lavazza.}
\end{quotation}

Per raggiungere tale obbiettivo,

\begin{quotation}
	\small  prima di tutto va escogitato il modo di
	risanare l'intelletto, e purificarlo, tanto quanto all'inizio è possibile, affinché comprenda felicemente,
	senza errore, e insomma nel migliore dei \\modi. \footnote{Trattato sull'emendazione dell'intelletto, traduzione e cura di Michele Lavazza.}
\end{quotation}

Ovvero, è importantissimo trovare un metodo di conoscenza non ingannevole e che distingua le cose vere da quelle dubbie. Il "Trattato sull'emendazione dell'intelletto" prosegue su questo argomento, pur rimanendo incompiuto.

In questa mia piccola opera, il tema della conoscenza viene approfondito nell'"Ethica more geometrico demonstrata", perciò a lì rimandiamo.

Prima di concludere questo capitolo, è curioso notare come Spinoza si dia una "morale provvisoria" da seguire, fino a che il suo metodo di conoscenza non sia consolidato: proprio come fece Cartesio mentre il dubbio iperbolico scuoteva le fondamenta della sua conoscenza.

\begin{quotation}
	\small \begin{enumerate}
		\item Parlare secondo le capacità del popolo, e mettere in atto tutti gli accorgimenti che evitino
		di ostacolare il raggiungimento del nostro scopo. Infatti da esso possiamo ottenere non pochi
		vantaggi, solo concedendo al suo livello di intendimento tutto ciò che è possibile; si aggiunga a
		questo che in tal modo essi presteranno volentieri orecchio per ascoltare la verità.
		\item Godere dei piaceri tanto quanto basta a preservare la salute.
		\item Infine, cercare di denaro, e di qualunque altra cosa, solo tanto quanto è sufficiente a
		supplire alle necessità della vita, della salute e dei costumi civili che non contrastano con il nostro
		scopo.\footnote{Trattato sull'emendazione dell'intelletto, traduzione e cura di Michele Lavazza.}
	\end{enumerate}
\end{quotation}

\newpage