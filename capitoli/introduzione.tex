
\chapter*{Introduzione}

Voglio raccontare del pensatore che tanto mi ha colpito dal mio avvicinamento alla filosofia: Barruch Spinoza. E' ironico e sbagliato iniziare questo resoconto con il verbo modale "volere": in fondo stiamo parlando del più grande pensatore determinista mai nato.\\
A ben pensarci, non sto scrivendo per scelta arbitraria, bensì per l'amore verso Spinoza: penso che approverebbe questo ragionamento.

Per arrivare fino a qua, ho svolto un percorso storico-filosofico che sta andando avanti, quindi forse incontrerò  altri pensatori che mi affascineranno altrettanto quanto Spinoza. Barruch mi è entrato nella testa più di Platone per la bellezza degli scritti, più degli stoici per il loro modo di vedere la vita (e a cui Spinoza gli è un po' debitore), più di Plotino per il fascino complicato della sua dottrina.

Riassumerò la filosofia di Spinoza, soffermandomi maggiormente sugli aspetti che  personalmente mi hanno colpito: quelli che si adattano alla vita nostra contemporanea, veramente tanti; sopratutto la sua umanità e il voler dimostrare che tutti gli uomini possono raggiungere la massima felicità, se vivono insieme e non l'uno contro l'altro. Spinoza è un filosofo attuale a quasi 350 anni dalla morte.
\newpage