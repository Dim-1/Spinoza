\chapter{II pensiero di Spinoza}
In questa parte mi soffermo sulla filosofia di Spinoza, sopratutto le conclusioni a cui giunge, mentre le spiegazioni saranno approfondire nella trattazione delle opere.
\subparagraph{La libertà:} una delle conclusioni più sconvolgenti a cui Spinoza arriva, è quella che, in ogni società, le nozioni di colpa, di merito e demerito, di bene e male, sono esclusivamente sociali e correlate all'obbedienza e alla disobbedienza. La migliore società  permetterà la libertà di pensare, esonerandola dal dover obbedire alla ragion di stato, che deve valere solo per le azioni.
\subparagraph{Perché il popolo è irrazionale e si vanta della propria schiavitù?} Spinoza denuncia il tradimento della natura e dell'umanità, in quanto l'uomo ha in odio la vita, se ne vergogna, fonda stati con schiavi, tiranni e preti impegnati a mutilare e soffocare la vita con leggi, autorità e doveri.\\
Il compito di Spinoza è proprio risvegliare la coscienza e la conoscenza, con il mezzo della dimostrazione geometrica. Capisce che una cosa diventa soggetta a giudizio etico quando viene sottratta all'ignoranza; è quindi nell'interesse dei potenti farci rimanere ignari delle cause delle cose, così che nessuno si ponga dubbi etici su pensieri e azioni. Per esempio, la questione vegetariana è diventata etica da quando abbiamo conoscenze ambientali, del regno animale, e mezzi superiori rispetto al passato per procurarci il cibo, per cui oggi possiamo porci il quesito se sia giusto oppure no mangiare animali. In passato questa domanda non si poneva, proprio perché mancava la conoscenza sugli animali, sul loro grado di sofferenza e sulle alternative alimentari.
\subparagraph{Il corpo:} nella sua dottrina del parallelismo (più avanti ne parleremo meglio), Spinoza nega ogni rapporto di causalità tra mente e corpo, ogni presunta superiorità della prima sul secondo, come era sempre stato. Quindi viene a cadere molta della morale cristiana che si fonda sul dominio delle passioni da parte dell'intelletto.\\
Ciò che è azione o passione nell'anima, è necessariamente azione o passione nel corpo. "L'oggetto dell'idea costituente la mente umana è il corpo", cioè "la mente umana è unita al corpo"\footnote{Ethica parte 2 proposizione 13 e suo scolio.}; mai si era avuta una tale valorizzazione del corpo umano, e un'affermazione così netta della sua unione con la mente.
\subparagraph{Conatus:} o appetito, è lo sforzo con cui ogni cosa cerca di perseverare nel suo essere, con cui non cerca solo di sopravvivere, bensì di migliorare la propria condizione, di realizzare al massimo grado la propria essenza; è desiderio nell'uomo quando ne siamo consapevoli. Se la cosa accresce il nostro essere, ci darà gioia (\textit{letizia}), in caso contrario tristezza. La nostra coscienza non è altro che il sentimento del passaggio da gioa a tristezza o viceversa, a testimoniare le variazioni del nostro appetito in funzione degli altri corpi o delle altre idee.\\
Quindi il desiderio di vita ci spinge a connetterci con altri corpi, definendosi a vicenda, per accrescere la nostra potenza di agire (il nostro essere).
\subparagraph{La volizione non esiste:}fintanto che abbiamo idee\footnote{Idea è un termine che nella storia della filosofia ha avuto un'evoluzione, acquisendo significati diversi: deriva dal termine greco "eidos", che significa "forma", in particolare forma ontologica, cioè essenza, ciò che è una certa cosa (risponde alla domanda socratica: "Che cosa è?); nel neoplatonismo e nella patristica cristiana, le idee diventano i pensieri del supremo intelletto, che fungono da archetipi per le cose del mondo; in epoca moderna, da Cartesio e ancor più da John Locke, con "idea" si indica un semplice contenuto della mente e del pensiero umano, "ciò che è oggetto dell'intelletto quando un uomo pensa"(John Locke, Saggio sull'intelletto umano).} inadeguate (non conosciamo le cause e i perché delle cose), in quanto esseri coscienti, comprendiamo solamente gli effetti di quello che ci circonda. Le condizioni all'interno delle quali conosciamo il mondo intorno a noi, ci fanno apparire le variazioni del nostro essere come effetti separati dalle loro proprie cause. Ci riteniamo liberi, dato che siamo consci solamente delle nostre volizioni e desideri, mentre le cause, da cui siamo disposti ad appetire e volere, nemmeno ce le "sogniamo".\\
La nostra coscienza va così colmando la propria ignoranza prendendo gli effetti per cause, così fa dell'idea di un certo effetto, la finalità delle sue proprie azioni. Quando non possiamo essere causa primaria, immaginiamo un Dio dotato di intelletto e volontà, operante secondo finalità, per preparare all'uomo un mondo a misura della sua gloria e dei suoi castighi.\\
La volontà è solamente il nostro sforzo consapevole per raggiungere ciò che desideriamo, ignorando le ragioni per cui questo nostro sforzo ci appare nostro desiderio. La vera causa che ci muove è il \textit{conatus}, che ci spinge ad accrescere il nostro essere, ma senza idee adeguate del mondo, senza comprendere le leggi necessarie che muovono la natura, pensiamo di avere libertà di scelta e viviamo vittime della fortuna (perché ignoriamo le cause).
\subparagraph{Il bene non esiste:}l'uomo non desidera il bene, bensì bene è ciò che desideriamo, che corrisponde a ciò che ci è utile. Quello che percepiamo come bene è semplicemente qualcosa che si compone con il nostro essere , che compiace il proprio istinto di conservazione (\textit{conatus}): ciò è detto buono e ci procura gioa, in caso contrario è cattivo (o malvagio) e causa tristezza.\\
Le tradizionali nozioni di bene/male, ordine/confusione, merito/peccato sono frutto dell'immaginazione, che crede le cose fatte per l'umanità da Dio: bene sarebbe ciò che comanda Dio e che giova la salute; ordinate sono le cose che possiamo facilmente ricordare.\\
Dal punto di vista della natura e di Dio, vi sono solamente dei rapporti che si compongono e decompongono secondo leggi necessari ed eterne. \textbf{Dio è al di là del bene e del male}. Il male esiste solamente nelle idee inadeguate, che non ci fanno cogliere la realtà per quella che è e ci causano un giudizio personale negativo, e nelle affezioni di tristezza che ne deriva, che diminuiscono il nostro essere. Chi ha sempre idee adeguate, comprendendo le regole della natura, compie solo azioni buone, che aumentano il proprio essere, non subisce passioni, che possono essere sia buone che cattive, dipendenti dal caso.
\subparagraph{L'etica sostituisce la morale:}questo è  temuto da molti ordini statali costituiti. La morale è trascendente (al di fuori di noi) ed è il giudizio di Dio, delle leggi  o delle tradizioni, che determinano l'opposizione dei valori bene/male. L'etica è immanente, è la discriminazione del buono dal cattivo (cioè di ciò che ci è utile oppure no), che può avvenire se siamo coscienti dei rapporti di natura. Quindi i tiranni, i controllori, hanno bisogno di questa contrasto bene/male per controllare le persone, è nei loro scopi mantenerci ignoranti, e quindi tristi, affinché non agiamo ma patiamo quello che ci viene da fuori, che è chiamato bene secondo la morale dell'autorità.\\
La morale coglie solo gli effetti e fraintende la natura, è sufficiente non comprendere per moralizzare. Per Spinoza la morale è pericolosa perché impedisce la formazione della conoscenza. Sull'aspetto dell'ignoranza il nostro filosofo insiste molto, in tutte le sue opere: solo conoscendo i rapporti di natura, ciò che ci spinge ad agire, possiamo aspirare al miglior stile di vita: come vedremo successivamente, Spinoza dimostra, con il suo rigore geometrico, che la cosa più buona per un uomo è l'ente che più gli assomiglia, cioè un altro uomo, e quindi l'intera umanità. Solo conoscendo approdiamo alla felicità nostra e della comunità.
\subparagraph{La religione è superstizione:}questo Dio degli uomini ignoranti, dotato di libertà umana, essendo immaginato alla stregua dell'indole di questi stessi uomini, fu giudicato agire ad uso degli uomini e, per legarli a sé, di essere onorato. Affinché Dio ci preferisse ad altri uomini, escogitiamo diversi modi di adorarlo, così da essere ricompensati. Da questo pregiudizio nasce la superstizione, che vuole spiegare con il massimo sforzo le cause finali del mondo. E questo Dio attribuisce giudizi che superano la comprensione umana, così da giustificare la nostra ignoranza; questa impotenza diventa il mezzo di assoggettamento dell'umanità, che cessa di chiedersi le cause delle cause, i perché dei perché, fino a rifugiarsi nella  \textbf{volontà di Dio, cioè l'asilo dell'ignoranza}.\\
Tolta la non conoscenza, vien meno lo stupore, l'unico mezzo che sostiene l'autorità.\\
Lo stato buono deve proporre ai cittadini l'amore delle libertà piuttosto che la speranza di ricompense per la loro buona condotta: come farà notare poi Nietzsche, conduciamo un simulacro di vita, non sogniamo che di evitare di morire, e tutta la nostra vita è culto della morte.
\subparagraph{Ancora sulla libertà:}come chiariremo meglio nel prima parte dell'Ethica, Spinoza vede l'universo costituito da un'unica sostanza che si concretizza nel mondo secondo leggi necessarie (che poi sono quelle che oggi vengono studiate dalla fisica, biologia, chimica, \dots). Questa sostanza è libera unicamente perché è \textit{causa sui}, non dipende da altro, ma è quel che è non per volizione, bensì genera il mondo perché non può essere altrimenti.\\
Allora, quando un uomo è libero?\\
Non quando riteniamo di volere una cosa non conoscendo le cause che ci spingono a tale desiderio; non siamo liberi di fare quel che si vuole, in quanto siamo indirizzati da desideri e istinti di sopravvivenza. Per Spinoza siamo liberi quando conosciamo le cause del nostro agire, e quindi possiamo essere ciò che si è, quando non si ha nessun impedimento di esprimere il nostro essere; conoscendo ed esprimendo la nostra essenza, raggiungiamo la felicità (cioè realizziamo le qualità proprie del nostro essere). \textbf{Libertà è consapevolezza di sé, è conoscenza}.
\subparagraph{Deus sive natura:}"Dio ossia la natura". Il Dio di Spinoza è visto come il dio degli scienziati (per esempio di Albert Einstein), proprio perché, se nell'Etica ogni volta che troviamo la parola Dio sostituiamo natura, quasi tutte le affermazioni del libro avrebbero senso a livello semantico. Infatti questo immanentismo di Dio è stato interpretato come ateismo da alcuni, anche se Spinoza ha sempre smentito questa accusa. Quello che fa del nostro un filosofo al passo con i tempi è proprio questa interpretazione, e il suo pensiero è coerente anche senza Dio, se per Dio intendiamo qualcosa di trascendente o dotato di volontà.\\
A Spinoza è bastata la natura con le sue leggi per spiegarci tutto, come va il mondo, i nostri impulsi e come render buona la nostra vita qui e adesso, non dopo la morte.
\newpage