\chapter*{II pensiero di Spinoza\footnote{In questa parte delineo la filosofia di Spinoza, sopratutto le conclusioni a cui giunge, mentre le spiegazioni saranno approfondire nella trattazione delle opere}}
\subparagraph{La libertà:} una delle conclusioni più sconvolgenti a cui Spinoza arriva, è che, in ogni società, le nozioni di colpa, di merito e demerito, di bene e male, sono esclusivamente sociali e correlate all'obbedienza e alla disobbedienza. La società migliore sarà quindi quella che permette la libertà di pensare, esonerandola dal dover obbedire alla ragion di stato, che deve valere solo per le azioni.
\subparagraph{Perché il popolo è irrazionale e si vanta della propria schiavitù?} Spinoza denuncia il tradimento della natura e dell'umanità, in quanto l'uomo ha in odio la vita, se ne vergogna, fonda stati con schiavi, tiranni e preti impegnati a mutilare e soffocare la vita con leggi, autorità e doveri. Il compito di Spinoza è proprio risvegliare la coscienza e la conoscenza, con il mezzo della dimostrazione geometrica. Capisce che una cosa diventa soggetta a giudizio etico quando viene sottratta all'ignoranza; è quindi nell'interesse dei potenti farci rimanere ignari delle cause delle cose, così che nessuno si ponga dubbi etici su pensieri e azioni\footnote{Per esempio, la questione vegetariana è diventata etica da quando abbiamo conoscenze ambientali, del regno animale, e mezzi superiori rispetto al passato per procurarci il cibo, per cui oggi possiamo porci il quesito se sia giusto oppure no mangiare animali. In passato questa domanda non si poneva, proprio perché mancava la conoscenza sugli animali, sul loro grado di sofferenza e sulle alternative alimentari.}.
\subparagraph{Il corpo:} nella sua dottrina del parallelismo, Spinoza nega ogni rapporto di causalità tra mente e corpo, ogni presunta superiorità della prima sul secondo, come era sempre stato. Quindi viene a cadere molta della morale cristiana che si fonda sul dominio delle passioni da parte dell'intelletto. Ciò che è azione o passione nell'anima, è necessariamente azione o passione nel corpo.
\subparagraph{Conatus:} o appetito, è lo sforzo con cui ogni cosa cerca di perseverare nel suo essere, con cui cerca di sopravvivere e migliorare la propria condizione; desiderio nell'uomo quando ne siamo consapevoli. Se la cosa accresce il nostro essere, ci darà gioia (\textit{letizia}), in caso contrario tristezza. La nostra coscienza non è altro che il sentimento del passaggio da gioa a tristezza o viceversa, a testimoniare le variazioni del nostro appetito in funzione degli altri corpi o delle altre idee.
\subparagraph{Contingenza:}fintanto che abbiamo idee inadeguate (non conosciamo le cause e i perché delle cose), in quanto esseri coscienti, comprendiamo solamente gli effetti di quello che ci circonda . Le condizioni all'interno delle quali conosciamo il mondo intorno a noi, ci fanno apparire le variazioni del nostro essere come effetti separati dalle loro proprie cause, e tutto ci sembra frutto del caso.

La nostra coscienza va così colmando la propria ignoranza, che genera impotenza, prendendo gli effetti per cause, così fa dell'idea di un certo effetto, la finalità delle sue proprie azioni. Quindi la coscienza si illude di avere il libero arbitrio, in quanto pensa di invocare il suo potere sui corpi. E quando non possiamo essere causa primaria, immaginiamo un Dio dotato di intelletto e volontà, operante secondo finalità, per preparare all'uomo un mondo a misura della sua gloria e dei suoi castighi.

La vera causa che ci muove è il \textit{conatus}, che ci spinge ad accrescere il nostro essere, ma senza idee adeguate del mondo, senza comprendere le leggi necessarie che muovono la natura, viviamo vittime della fortuna.