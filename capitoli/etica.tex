\chapter{Ethica more geometrico demonstrata}
Etica dimostrata con metodo geometrico.\\
L'obbiettivo dell'Etica è:\begin{itemize}
	\item Come arrivare ad avere passioni sempre gioiose.
	\item Come avere sempre idee adeguate, da cui derivavano proprio i sentimenti buoni.
	\item Come divenire coscienti di se stessi, di Dio e delle cose.
\end{itemize}
Tutte le teorie dell'Etica (sostanza, attributi, parallelismo, \dots) non sono separabili da queste tre grandi tesi pratiche.

L'Etica sembra scritto due volte: la prima ispirandosi agli "Elementi" di Euclide \footnote{che ha gettato le basi della geometria nel 300 a.c. circa.}, con assiomi, definizioni, postulati, proposizioni e dimostrazioni. Qui Spinoza cerca di dimostrare, con metodi logici e deduttivi simili appunto ai teoremi geometrici, prima quale sia la realtà, poi cosa sia l'uomo, la mente, e infine la via per realizzare la propria natura ed esser felice. La seconda volta, che si mischia e compenetra la prima, è l'intreccio dei vari scolii (commenti) alle varie proposizioni, che illustrano le tesi pratiche a cui giunge in modo veramente appassionato, dove Spinoza si getta con tutto il suo cuore contro le ipocrisie del mondo. Alcune parti dell'Etica, per esempio l'appendice al primo libro, sono di una chiarezza e sincerità stupefacenti, e il loro contenuto meriterebbe di esser letto in tutte le scuole.
\section{Prima parte}