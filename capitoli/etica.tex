\chapter{Ethica more geometrico demonstrata}
Etica dimostrata con metodo geometrico.\\
L'obbiettivo dell'Etica è:\begin{itemize}
	\item Come arrivare ad avere passioni sempre gioiose.
	\item Come avere sempre idee adeguate, da cui derivavano proprio i sentimenti buoni.
	\item Come divenire coscienti di se stessi, di Dio e delle cose.
\end{itemize}
Tutte le teorie dell'Etica (sostanza, attributi, parallelismo, \dots) non sono separabili da queste tre grandi tesi pratiche.

L'Etica sembra scritto due volte: la prima ispirandosi agli "Elementi" di Euclide \footnote{che ha gettato le basi della geometria nel 300 a.c. circa.}, con assiomi, definizioni, postulati, proposizioni e dimostrazioni. Qui Spinoza cerca di dimostrare, con metodi logici e deduttivi simili appunto ai teoremi geometrici, prima quale sia la realtà, poi cosa sia l'uomo, la mente, e infine la via per realizzare la propria natura ed esser felice; l'andamento dimostrativo è dovuto alla convinzione che la natura si organizzi secondo regole matematiche, come già avevano affermato Galileo e Cartesio. La seconda volta, che si mischia e compenetra la prima, è l'intreccio dei vari scolii (commenti) alle varie proposizioni, che illustrano le tesi pratiche a cui giunge in modo veramente appassionato, dove Spinoza si getta con tutto il suo cuore contro le ipocrisie del mondo. Alcune parti dell'Etica, per esempio l'appendice al primo libro, sono di una chiarezza e sincerità stupefacenti, e il loro contenuto meriterebbe di esser letto in tutte le scuole.


\section{Prima parte - Dio}
\begin{quotation}
\small	Per sostanza intendo ciò che è in sé e per sé si concepisce: ossia ciò il cui concetto non ha bisogno di essere formato dal concetto di altro.\footnote{Etica prima parte definizione 3.}
\end{quotation}
Ovvero la sostanza non ha bisogno di altro per esistere ed essere concepita, è autonoma da un punto di vista ontologico e logico. Procedendo tra le prime definizioni e proposizioni, Spinoza dimostra che la sostanza è increata, ingenerata, infinita, eterna e autonoma (non ha bisogno di altro per esistere); è ovunque e unica; ed è necessaria, ovvero la sua essenza implica la sua esistenza, non può essere in nessun altro modo.
Questa sostanza unica è la realtà in toto, è Dio, è la natura. 
Ed ha infiniti attributi, ognuno distinto dall'altro (sono concepiti senza il bisogno di altri attributi), che esprimono qualità semplici della sostanza, la sua essenza.
\begin{quotation}
	\small Per attributo intendo ciò, che l'intelletto percepisce, come costituente la sua essenza(della sostanza.)\footnote{Etica prima parte definizione 4.}
\end{quotation}
L'uomo conosce solo due attributi: il pensiero e l'estensione. I corpi implicano l'estensione (o le menti il pensiero) sotto la medesima forma in cui l'estensione (o il pensiero) è un attributo della sostanza divina; ovvero Dio non possiede le stesse qualità del mondo in forma superiore o altra forma, ma nella stessa modalità.
\begin{quotation}
	\small Per modo intendo le affezioni della sostanza, ossia ciò, che è in altro, per cui anche viene concepito.\footnote{Etica parte prima definizione 5.}
\end{quotation}
I modi non sono in sé (solo la sostanza esiste in sé), ma in altro, cioè derivano dalla sostanza: sono la concretizzazione degli attributi della sostanza, sono i corpi se si guarda all'estensione, le idee riferendosi al pensiero, ognuno prodotti in quegli stessi attributi che costituiscano l'essenza della sostanza.

I medesimi attributi esprimono qualità della sostanza che essi compongano e dei modi che contengono (corpi e idee): cioè Dio, la natura è in tutte le cose, \textbf{è immanente}.

Dio è causa libera, perché agisce per la sola necessità della sua natura e per la sola necessità della sua natura esiste: è la \textbf{natura naturante}; invece \textbf{natura naturata} è "tutto ciò che segue dalla necessità della natura di Dio, ossia di ogni attributo di Dio, tutti i modi degli attributi di Dio, in quanto si considerano come cose che sono in Dio, e che senza Dio non possono ne essere ne essere concepite"\footnote{Etica parte prima scolio alla proposizione 29.}.

Detto in termini più moderni, terreni e forse non corretti a livello filologico: i corpi intorno a noi e i contenuti della nostra mente sono le leggi della natura che si esplicano, necessariamente, cioè in un solo modo possibile, esattamente come una legge fisica, chimica o biologica. Tutta l'esistenza è necessariamente determinata da un'infinita catena di relazioni causa-effetto, secondo le regole di natura. 

Quindi non vi è nulla di contingente in natura, che sarebbe potuto non essere, cioè le esistenze non sono prodotte dall'atto di una volontà divina, bensì non poterono essere prodotte da Dio in nessuna altra maniera ne ordine diverso da come sono state prodotte\footnote{Etica prima parte corollario 2 alla proposizione 32.}.


Spinoza afferma: la volontà ha bisogno di una causa, come tutte le altre cose, da cui sia determinata a esistere e operare in un certo modo; quindi tra intelletto e volontà non sussiste alcuna differenza. Nell'appendice alla parte prima dell'Etica il nostro si lancia contro il più grande pregiudizio dell'umanità: supporre che agiamo per un fine, che siamo volitivi.
\begin{quotation}
	\small E poiché
	tutti i pregiudizi che m’accingo a sottoporre ad esame dipendono da quest’unico, che gli
	umani immaginano comunemente che le cose della natura operino, come essi stessi fanno,
	mirando a uno scopo (addirittura essi danno per certo che Dio stesso diriga le cose a un fine determinato: avendo egli fatto ogni cosa a pro dell’Uomo, e avendo fatto l’Uomo per essere da lui adorato), io prenderò dapprima in considerazione questo solo pregiudizio; e cercherò di scoprire, per cominciare, la causa per cui la maggioranza degli umani se ne sta
	tranquilla in questo pregiudizio, e la totalità è per natura così propensa ad accettarlo; mostrerò poi la falsità del pregiudizio; e infine mostrerò come dal pregiudizio stesso
	siano sorti gli altri pregiudizi che concernono il bene e il male, il merito e il peccato, la lode
	e il biasimo, l’ordine e il disordine, la bellezza e la bruttezza, e via dicendo. Non è questo il
	luogo per mostrare come tali pregiudizi derivino dalla natura della mente umana: qui basterà riconoscere - ed io lo prenderò come fondamento - ciò tutti debbono ammettere: cioè
	che tutti gli umani nascono ignorando le cause delle cose, e tutti sono portati istintivamente a cercare il loro utile, e di questo hanno coscienza. Di qui derivano alcune conseguenze.
	1°, Gli umani sono convinti di essere liberi perché sono consapevoli delle loro volizioni e
	dei loro desideri istintivi e perché non pensano neanche in sogno dato che ne sono ignari -
	alle cause che li orientano a desiderare e a volere. 2°, Gli umani agiscono in ogni caso in vista di un fine, cioè in vista dell’utile che appetiscono: e ne deriva che essi si preoccupino
	sempre di conoscere soltanto le cause finali di ciò hanno compiuto, e , quando le abbiano
	apprese, smettano di preoccuparsi: e questo è ragionevole, poiché a questo punto non hanno motivo di porsi altri dubbi. (Non avendo nessuno che gli dia spiegazioni corrette, perché
	tutti si trovano nelle stesse condizioni, gli umani sono costretti a prendere se stessi come
	esemplare e a riflettere sui fini che di solito spingono ciascuno a compiere le azioni più comuni: e in questo modo col metro del loro sentimento misurano tutto il resto della natura).
	D’altronde gli umani trovano in se stessi, e all’esterno di sé, troppi mezzi assai efficaci per
	conseguire il loro utile - quali gli occhi per vedere, i denti per masticare, i vegetali e gli animali per nutrirsi, il sole che li illumina, il mare che alimenta per loro i pesci - perché essi
	non considerino da sempre, spontaneamente, tutte le cose della natura come mezzi per raggiungere il loro utile; e poiché sanno di non aver essi stessi apprestato quei mezzi, ma di
	averli trovati, ne hanno tratto il motivo per credere che ci sia qualcuno, estraneo alla specie umana, che abbia apprestato quei mezzi per loro uso. Dopo avere scoperto nelle cose la
	qualità di mezzi, gli umani non hanno, evidentemente, potuto credere che quelle cose si
	siano fatte da sé; e, tenendo conto di come essi si apprestano i mezzi di cui hanno bisogno,
	hanno dovuto concludere che esistano uno, o più, reggitori della natura, forniti di libertà
	come gli umani, che hanno disposto a favore degli umani tutte le cose e le hanno tutte destinate al loro uso. E anche il sentimento di quei reggitori - del quale essi non hanno mai
	avuto notizia diretta - gli umani hanno dovuto immaginare in base al proprio: ed hanno così stabilito che gli Dei dirigono tutte le cose per uso degli umani, così da legarseli e da esser
	tenuti da loro nel massimo onore; e di qui poi ognuno ha escogitato, secondo il suo modo
	di vedere, i diversi modi di render culto a Dio, così da essere amato da Dio più gli altri e da
	meritare che Dio rivolga l’intera natura a pro della sua cieca cupidigia e della sua insaziabile avidità. E questo pregiudizio, diventato superstizione, s’è profondamente radicato nelle
	menti: ed è stato la causa per cui tutti si sono dedicati con ogni impegno a capire e a spiegare le cause finali di tutte le cose. Ma si direbbe che questo cercar di mostrare che la natura
	non fa nulla invano (cioè nulla non che sia utile agli umani) è riuscito a mostrare soltanto
	che la stessa follia che è negli umani è anche nella natura e negli Dei. Vediamo un po’ a
	qual punto la cosa è arrivata. Fra i tanti vantaggi offerti dalla natura i ricercatori hanno dovuto trovare non poche cose svantaggiose, quali tempeste, terremoti, malattie eccetera: e
	hanno stabilito che questo si verifica perché gli Dei sono irati a causa di offese recate loro
	dagli umani o di scorrettezze commesse nel culto; e sebbene l’esperienza quotidiana affermi a gran voce e mostri con infiniti esempi che fortune e sfortune toccano nella stessa maniera e indistintamente ai pii e agli empi, quei ricercatori non hanno dimesso il pregiudizio
	ormai inveterato, giudicando che porre quella incomprensibile uniformità fra le altre cose
	ignote, delle quali non si conosce il perché, e conservare così la loro presente e innata condizione di ignoranza, sia più facile che demolire tutte quelle loro costruzioni e concepirne
	un’altra, nuova: e su una tale base hanno decretato, come cosa certa, che le risoluzioni degli Dei superano di gran lunga il comprendonio umano. Questo trovato, da solo, sarebbe
	stato sufficiente a che la verità restasse in eterno nascosta al genere umano, se la Matematica – che si occupa non dei fini, ma delle essenze e delle proprietà delle figure - non avesse
	mostrato agli umani un altro criterio di verità; e oltre alla Matematica si può indicare, senza che sia necessario enumerarli qui, altri fattori, grazie ai quali ha potuto accadere che taluni umani si siano accorti della natura di pregiudizio che hanno queste credenze comuni e
	siano riusciti a giungere alla vera conoscenza delle cose.
	Con quanto precede ho spiegato a sufficienza ciò che mi ero proposto come primo punto.
	Per mostrare ora che la natura non ha alcun fine che le sia stato prefissato, e che tutte le
	cause finali non sono invenzioni umane, non ci vuol molto. Credo infatti che questo risulti
	chiaro tanto tenendo conto dei fondamenti e delle cause dai quali ho mostrato che il pregiudizio in parola ha tratto origine, quanto rammentando la Proposizione 16 e le Conseguenze della Prop. 32, e inoltre tutte le altre proposizioni, con le quali ho mostrato che nella natura tutto è prodotto ed accade per una certa necessità eterna e con una perfezione suprema. Aggiungerò tuttavia ancora un’osservazione: che questa dottrina dei fini sconvolge
	completamente la natura. Essa infatti considera come effetto ciò che invero è causa, e viceversa; poi mette dopo ciò che per natura è prima; e infine riduce imperfettissimo ciò che
	per natura è supremo e perfettissimo. Lasciamo da parte i primi due punti, che sono evidenti di per sé. Quanto al terzo, come risulta dalle Proposizioni 21, 22, 23, è perfettissimo
	quell’effetto che è prodotto da Dio immediatamente, ed una cosa è tanto più imperfetta
	quante più sono le cause intermedie di cui essa ha bisogno per essere prodotta: ma se le cose che sono state prodotte immediatamente da Dio fossero state fatte perché Dio conseguisse un suo fine, allora le ultime - a causa delle quali sono state fatte le prime - sarebbero
	necessariamente le più eccellenti. Inoltre, questa dottrina annienta la perfezione di Dio: dato che necessariamente, se agisce in vista di un fine, Dio manca di qualcosa, che desidera e
	cerca. E quantunque i teologi e i metafisici distinguano tra fine di indigenza (Dio creerebbe
	le cose perché ne ha bisogno) e fine di assimilazione (Dio vuole le cose siano per attribuire
	ad esse la sua beatitudine), essi tuttavia confessano che Dio ha fatto tutte le cose per se
	stesso, non per le creature: essi infatti non possono trovare che prima della creazione ci
	fosse un qualche Ente, oltre a Dio, a causa del quale Dio operasse; e pertanto debbono necessariamente ammettere che Dio mancava delle cose di cui ha predisposto l’esistenza, e le
	desiderava: come è evidente da sé. Non si deve poi passar sotto silenzio che i seguaci di
	questa dottrina, i quali con l’individuare i fini delle cose hanno voluto mettere in mostra il
	loro ingegno, hanno - per rendere plausibili le loro affermazioni - escogitato una nuova maniera di argomentare: la riduzione non all’impossibile, ma all’ignoranza: e questo mostra
	che per sostenere la loro dottrina non c’era alcun argomento vero. Per fare un esempio, se
	una tegola è caduta da un tetto sulla testa di qualcuno e l’ha ucciso, essi dimostrano nel
	modo seguente che la tegola è caduta per uccidere quell’uomo. Se la tegola non è caduta
	per volontà di Dio al fine predetto, chiederanno, come mai tante circostanze (perché spesso
	sono molte a concorrere) hanno potuto concorrere casualmente? Qualcuno risponderà che
	il caso avvenne perché tirava vento e perché l’uomo aveva bisogno di passare di là. Ed essi
	diranno: e perché il vento soffiò proprio allora? e perché quell’uomo doveva passare di là
	proprio nello stesso tempo? Qualcuno replicherà che il vento s’era levato proprio allora
	perché il giorno precedente, mentre il tempo era ancora calmo, il mare aveva cominciato
	ad agitarsi; e l’uomo era stato invitato da un amico. Ed essi di nuovo - perché si può domandare all’infinito: perché il mare era mosso? perché l’uomo era stato invitato in quel
	momento? E non smetteranno di chiedere le cause delle cause fin che l’interlocutore non si
	rifugerà nella volontà di Dio, cioè nel ricovero dell’ignoranza. Per fare un altro esempio, i
	seguaci della dottrina dei fini stupiscono quando si pongono a considerare la struttura del
	corpo umano: e, siccome ignorano le cause di un così mirabile meccanismo, concludono
	che esso non s’è costruito da sé per certe sue leggi intrinseche, ma è il prodotto di un’arte
	divina o soprannaturale, dalla quale esso è stato congegnato in maniera che un pezzo non
	danneggi l’altro, o, piuttosto, che ogni pezzo cooperi con ogni altro. Vigendo tali criteri accade che chi vuol conoscere le vere cause degli eventi miracolosi, come chi cerca di capire
	da scienziato le cose della natura e non di meravigliarsene da sciocco, sia in generale giudicato eretico ed empio e proclamato tale da coloro che il volgo venera come interpreti della
	natura e degli Dei. Costoro sanno infatti che eliminando l’ignoranza si distrugge anche lo
	stupore, cioè l’unico mezzo che essi hanno di conservare credibile e di salvaguardare la loro
	autorità. Ma ora lascio questo argomento per passare a quello che ho stabilito di trattare in
	terzo luogo.
	Essendosi persuasi che tutto ciò che accade è finalizzato a loro, gli umani hanno dovuto
	arrivar a giudicare che in ogni cosa il più importante è ciò che è più utile a loro, e che le cose più eccellenti sono quelle che danno a loro maggior piacere. Su questa base essi hanno,
	logicamente, dovuto formare le nozioni con le quali potere spiegare la natura delle cose:
	cioè le nozioni di Bene, di Male, di Ordine, di Confusione, di Caldo, di Freddo, di Bellezza,
	di Bruttezza; e dalla convinzione di esser liberi, che essi hanno, sono poi sorte le nozioni di
	Lode e di Biasimo, di Peccato e di Merito. Di queste ultime nozioni mi occuperò più avanti,
	dopo avere trattato della natura umana; qui invece spiegherò brevemente le prime. Gli
	umani dunque hanno chiamato Bene tutto ciò che favorisce la salute e inclina al culto di
	Dio, e Male ciò che è contrario a queste cose. Essi, poiché non penetrano intellettualmente
	la natura delle cose, ma si limitano all’apparenza di esse, che colpisce la loro immaginazione, non possono - prendendo l’immaginazione per l’intelletto - esprimere sulle cose giudizi
	corrispondenti al vero; e così, ignari della natura delle cose, e anche della natura propria,
	credono fermamente che nelle cose ci sia un ordine. Infatti, quando determinate cose sono
	disposte in maniera tale che noi, dopo averle considerate, possiamo facilmente figurarcele
	nella mente e quindi facilmente ricordarle, noi le diciamo bene ordinate; quando invece accade il contrario noi diciamo quelle cose male ordinate o confuse. E poiché le cose che noi
	immaginiamo facilmente ci piacciono più delle altre, gli umani preferiscono l’ordine alla
	confusione - come se l’ordine della natura fosse non qualcosa che vi scopre la nostra immaginazione, ma una realtà: e dicono che Dio ha creato le cose con ordine, attribuendo con
	ciò a Dio, senza saperlo, un’immaginazione, o magari convincendosi che Dio, a favore
	dell’immaginazione umana, abbia disposto le cose in modo da poter essere immaginate con
	la maggior agevolezza; e forse, avviati gli umani su questa strada, non li tratterrà il riflettere che ci sono infinite cose che superano di parecchio la nostra immaginazione, e molte che
	la confondono, debole com’è. Ma di questo ho detto abbastanza. Per quanto concerne le al-
	tre nozioni, anche esse non sono altro che modi dell’immaginare, dai quali l’immaginazione
	è variamente interessata: ma gl’ignoranti le considerano attributi principali delle cose, dato
	che, come abbiamo già detto, essi credono che tutte le cose siano state prodotte in vista di
	loro stessi, e chiamano le cose buone o cattive, sane o guaste o marce, a seconda del modo
	in cui ne sono toccati. Per esempio, se la sollecitazione che arriva ai nervi dagli oggetti percepiti attraverso gli occhi procura un senso di benessere, gli oggetti che ne sono causa sono
	chiamati belli; gli oggetti da cui proviene una sollecitazione sgradevole sono chiamati brutti. Gli oggetti poi che sollecitano i nervi tramite l’odorato sono, a loro volta, detti profumati
	o maleodoranti; quelli che sono percepiti dalla lingua sono detti dolci o amari, saporiti o insipidi; quelli che sono percepiti dal tatto sono detti duri o molli, ruvidi o lisci; di quelli, infine, che sollecitano i nervi per il tramite degli orecchi, si dice che producono un rumore, o
	un suono, o un’armonia. A proposito di quest’ultimo caso la follia degli umani è arrivata al
	punto di credere che dell’armonia si diletti anche Dio; e nemmeno mancano filosofi profondamente convinti che i movimenti dei corpi celesti producano un’armonia. Tutti questi fatti mostrano a sufficienza che sulle cose ciascuno ha espresso giudizi conformi alle caratteristiche del suo cervello, o, meglio, che la gente ha preso, in luogo delle cose, ciò che la sua
	immaginazione risentiva delle cose stesse. Per questo motivo non c’è da meravigliarsi (notiamo di passaggio anche questo) che tra gli umani siano sorte tutte le controversie filosofiche che conosciamo così bene, e che da esse sia infine uscito lo Scetticismo. Le strutture dei diversi corpi umani sono simili in molti aspetti, ma sono dissimili in moltissimi altri: e per questo ciò che a uno pare buono, a un altro pare cattivo; quel che per uno è
	ordinato, per un altro è confuso; quel che a uno fa piacere, a un altro fa dispiacere. Potrei
	continuare, ma mi fermo qui, sia perché non è questa la sede per diffondersi su un tale argomento, sia perché tutti ne hanno fatto sufficiente esperienza: tutti infatti sanno che
	quante teste, tanti pareri; che ognuno stima d’aver giudizio anche più del necessario; che
	ci son tante differenze fra le idee quante fra i gusti: detti, questi, che mostrano a sufficienza come gli umani giudichino delle cose secondo la disposizione del loro cervello, e come le
	immaginino più che comprenderle. Se infatti gli umani le comprendessero mediante l’intelletto, le cose nella loro realtà - come testimonia la Matematica - potrebbero magari non
	attrarre tutti, ma almeno convincere tutti alla stessa maniera.
	È dunque evidente che tutte le nozioni con le quali la gente è usa a "spiegare" la natura
	non sono altro che modi dell’immaginazione, e non chiariscono la struttura interna di alcunché ma soltanto ci informano sulla costituzione dell’immaginazione; e poiché questi enti hanno dei nomi, come se si trattasse di realtà esistenti fuori dell’immaginazione, io li
	chiamo enti non di ragione, ma d’immaginazione; e così è facile confutare tutti gli argomenti che vengono tratti da quelle nozioni contro il nostro modo di vedere. Molti infatti so-gliono argomentare così: Se tutte le cose sono uscite dalla necessità della perfettissima natura di Dio, di dove provengono dunque alla natura tante imperfezioni: le cose che si guastano fino a puzzare, le cose tanto brutte da suscitare la nausea, il disordine, il male, il peccato, eccetera? Ma, l’ho detto or ora, è facile confutare quei tali. La perfezione delle cose,
	infatti, si deve valutare solo in riguardo della loro natura e della loro potenza; e le cose non
	sono più o meno perfette a seconda che dilettano o urtano i sensi degli umani, a seconda
	che sono gradite alla natura umana o ad essa ripugnano.\footnote{Etica prima parte appendice.}
\end{quotation}
\section{Seconda parte - La mente umana}