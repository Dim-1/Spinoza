\chapter[Etica]{Ethica more geometrico demonstrata}
Etica dimostrata con metodo geometrico.\\L'obbiettivo dell'Etica è:\begin{itemize}
	\item Come arrivare ad avere passioni sempre gioiose.
	\item Come avere sempre idee adeguate, da cui derivavano proprio i sentimenti buoni.
	\item Come divenire coscienti di se stessi, di Dio e delle cose.
\end{itemize}

Tutte le teorie dell'Etica (sostanza, attributi, parallelismo, \dots) non sono separabili da queste tre grandi tesi pratiche.

L'Etica sembra scritto due volte: la prima ispirandosi agli "Elementi" di Euclide \footnote{che ha gettato le basi della geometria nel 300 a.c. circa.}, con assiomi, definizioni, postulati, proposizioni e dimostrazioni. Qui Spinoza cerca di dimostrare, con metodi logici e deduttivi simili appunto ai teoremi geometrici, prima quale sia la realtà, poi cosa sia l'uomo, la mente, e infine la via per realizzare la propria natura ed esser felice; l'andamento dimostrativo è dovuto alla convinzione che la natura si organizzi secondo regole matematiche, come già avevano affermato Galileo e Cartesio. La seconda volta, che si mischia e compenetra la prima, è l'intreccio degli scolii (commenti) alle varie proposizioni, che illustrano le tesi pratiche a cui giunge in modo veramente appassionato, dove Spinoza si getta con tutto il suo cuore contro le ipocrisie del mondo. Alcune parti dell'Etica, per esempio l'appendice al primo libro, sono di una chiarezza e sincerità stupefacenti, e il loro contenuto meriterebbe di esser letto in tutte le scuole.
\newpage

\section{Prima parte - Dio}
\begin{quotation}
\small	Per sostanza intendo ciò che è in sé e per sé si concepisce: ossia ciò il cui concetto non ha bisogno di essere formato dal concetto di altro.\footnote{Etica prima parte definizione 3.}
\end{quotation}

Ovvero la sostanza non ha bisogno di altro per esistere ed essere concepita, è autonoma da un punto di vista ontologico e logico. Procedendo tra le prime definizioni e proposizioni, Spinoza dimostra che la sostanza è increata, ingenerata, infinita, eterna e autonoma (non ha bisogno di altro per esistere); è ovunque e unica; ed è necessaria, ovvero la sua essenza implica la sua esistenza, non può essere in nessun altro modo.
Questa sostanza unica è la realtà in toto, è Dio, è la natura. 
Ed ha infiniti attributi, ognuno distinto dall'altro (sono concepiti senza il bisogno di altri attributi), che esprimono qualità semplici della sostanza, la sua essenza.

\begin{quotation}
	\small Per attributo intendo ciò, che l'intelletto percepisce, come costituente la sua essenza(della sostanza.)\footnote{Etica prima parte definizione 4.}
\end{quotation}

L'uomo conosce solo due attributi: il pensiero e l'estensione. I corpi implicano l'estensione (o le menti il pensiero) sotto la medesima forma in cui l'estensione (o il pensiero) è un attributo della sostanza divina; ovvero Dio non possiede le stesse qualità del mondo in forma superiore o altra forma, ma nella stessa modalità di qualsiasi ente.

\begin{quotation}
	\small Per modo intendo le affezioni della sostanza, ossia ciò, che è in altro, per cui anche viene concepito.\footnote{Etica parte prima definizione 5.}
\end{quotation}

I modi non sono in sé (solo la sostanza esiste in sé), ma in altro, cioè derivano dalla sostanza: sono la concretizzazione degli attributi della sostanza, sono i corpi se si guarda all'estensione, le idee riferendosi al pensiero, ognuno prodotti in quegli stessi attributi che costituiscano l'essenza della sostanza. Per essere precisi, dall'attributo dell'estensione derivano i modi infiniti di \textit{quiete} e \textit{moto}, dall'attributo del pensiero deriva il modo infinito dell'\textit{intelletto}: da questi si passa ai modi finiti, corpi e idee, in modo non molto chiaro, tanto che per alcuni autori ciò è vista come un'aporia.

I medesimi attributi esprimono qualità della sostanza che essi compongano e dei modi che contengono (corpi e idee): cioè Dio, la natura è in tutte le cose, \textbf{è immanente}.

Dio è causa libera, perché agisce per la sola necessità della sua natura e per la sola necessità della sua natura esiste: è la \textbf{natura naturante}; invece \textbf{natura naturata} è 
\begin{quotation}
	\small tutto ciò che segue dalla necessità della natura di Dio, ossia di ogni attributo di Dio, tutti i modi degli attributi di Dio, in quanto si considerano come cose che sono in Dio, e che senza Dio non possono ne essere ne essere concepite.\footnote{Etica parte prima scolio alla proposizione 29.}
\end{quotation}

Detto in termini più moderni, terreni e forse non corretti a livello filologico: i corpi intorno a noi e i contenuti della nostra mente sono le leggi della natura che si esplicano, necessariamente, cioè in un solo modo possibile, esattamente come una legge fisica, chimica o biologica. Tutta l'esistenza è necessariamente determinata da un'infinita catena di relazioni causa-effetto, secondo le regole di natura. 

Quindi non vi è nulla di contingente, che sarebbe potuto non essere, cioè le esistenze non sono prodotte dall'atto di una volontà divina, bensì non poterono essere prodotte da Dio in nessuna altra maniera ne ordine diverso da come sono state prodotte\footnote{Etica prima parte corollario 2 alla proposizione 32.}.


Spinoza afferma: la volontà ha bisogno di una causa, come tutte le altre cose, da cui sia determinata a esistere e operare in un certo modo; quindi tra intelletto e volontà non sussiste alcuna differenza. Nell'appendice alla parte prima dell'Etica il nostro si lancia contro il più grande pregiudizio dell'umanità: supporre che agiamo per un fine, che siamo volitivi.
\begin{quotation}
	\small E poiché
	tutti i pregiudizi che m’accingo a sottoporre ad esame dipendono da quest’unico, che gli
	umani immaginano comunemente che le cose della natura operino, come essi stessi fanno,
	mirando a uno scopo (addirittura essi danno per certo che Dio stesso diriga le cose a un fine determinato: avendo egli fatto ogni cosa a pro dell’Uomo, e avendo fatto l’Uomo per essere da lui adorato), io prenderò dapprima in considerazione questo solo pregiudizio; e cercherò di scoprire, per cominciare, la causa per cui la maggioranza degli umani se ne sta
	tranquilla in questo pregiudizio, e la totalità è per natura così propensa ad accettarlo; mostrerò poi la falsità del pregiudizio; e infine mostrerò come dal pregiudizio stesso
	siano sorti gli altri pregiudizi che concernono il bene e il male, il merito e il peccato, la lode
	e il biasimo, l’ordine e il disordine, la bellezza e la bruttezza, e via dicendo. Non è questo il
	luogo per mostrare come tali pregiudizi derivino dalla natura della mente umana: qui basterà riconoscere - ed io lo prenderò come fondamento - ciò tutti debbono ammettere: cioè
	che tutti gli umani nascono ignorando le cause delle cose, e tutti sono portati istintivamente a cercare il loro utile, e di questo hanno coscienza. Di qui derivano alcune conseguenze:
	\begin{enumerate}
		\item Gli umani sono convinti di essere liberi perché sono consapevoli delle loro volizioni e
		dei loro desideri istintivi e perché non pensano neanche in sogno dato che ne sono ignari -
		alle cause che li orientano a desiderare e a volere.
		\item Gli umani agiscono in ogni caso in vista di un fine, cioè in vista dell’utile che appetiscono: e ne deriva che essi si preoccupino
		sempre di conoscere soltanto le cause finali di ciò hanno compiuto, e , quando le abbiano
		apprese, smettano di preoccuparsi: e questo è ragionevole, poiché a questo punto non hanno motivo di porsi altri dubbi. (Non avendo nessuno che gli dia spiegazioni corrette, perché
		tutti si trovano nelle stesse condizioni, gli umani sono costretti a prendere se stessi come
		esemplare e a riflettere sui fini che di solito spingono ciascuno a compiere le azioni più comuni: e in questo modo col metro del loro sentimento misurano tutto il resto della natura).
	\end{enumerate} 
	D’altronde gli umani trovano in se stessi, e all’esterno di sé, troppi mezzi assai efficaci per
	conseguire il loro utile - quali gli occhi per vedere, i denti per masticare, i vegetali e gli animali per nutrirsi, il sole che li illumina, il mare che alimenta per loro i pesci - perché essi
	non considerino da sempre, spontaneamente, tutte le cose della natura come mezzi per raggiungere il loro utile; e poiché sanno di non aver essi stessi apprestato quei mezzi, ma di
	averli trovati, ne hanno tratto il motivo per credere che ci sia qualcuno, estraneo alla specie umana, che abbia apprestato quei mezzi per loro uso. Dopo avere scoperto nelle cose la
	qualità di mezzi, gli umani non hanno, evidentemente, potuto credere che quelle cose si
	siano fatte da sé; e, tenendo conto di come essi si apprestano i mezzi di cui hanno bisogno,
	hanno dovuto concludere che esistano uno, o più, reggitori della natura, forniti di libertà
	come gli umani, che hanno disposto a favore degli umani tutte le cose e le hanno tutte destinate al loro uso. E anche il sentimento di quei reggitori - del quale essi non hanno mai
	avuto notizia diretta - gli umani hanno dovuto immaginare in base al proprio: ed hanno così stabilito che gli Dei dirigono tutte le cose per uso degli umani, così da legarseli e da esser
	tenuti da loro nel massimo onore; e di qui poi ognuno ha escogitato, secondo il suo modo
	di vedere, i diversi modi di render culto a Dio, così da essere amato da Dio più gli altri e da
	meritare che Dio rivolga l’intera natura a pro della sua cieca cupidigia e della sua insaziabile avidità. E questo pregiudizio, diventato superstizione, s’è profondamente radicato nelle
	menti: ed è stato la causa per cui tutti si sono dedicati con ogni impegno a capire e a spiegare le cause finali di tutte le cose. Ma si direbbe che questo cercar di mostrare che la natura
	non fa nulla invano (cioè nulla non che sia utile agli umani) è riuscito a mostrare soltanto
	che la stessa follia che è negli umani è anche nella natura e negli Dei. Vediamo un po’ a
	qual punto la cosa è arrivata. Fra i tanti vantaggi offerti dalla natura i ricercatori hanno dovuto trovare non poche cose svantaggiose, quali tempeste, terremoti, malattie eccetera: e
	hanno stabilito che questo si verifica perché gli Dei sono irati a causa di offese recate loro
	dagli umani o di scorrettezze commesse nel culto; \textbf{e sebbene l’esperienza quotidiana affermi a gran voce e mostri con infiniti esempi che fortune e sfortune toccano nella stessa maniera e indistintamente ai pii e agli empi, quei ricercatori non hanno dimesso il pregiudizio
	ormai radicato, giudicando di collocare quella incomprensibile uniformità fra le cose
	ignote, delle quali non si conosce il perché, e conservare così la loro presente e innata condizione di ignoranza, sia più facile che demolire tutte quelle loro costruzioni e concepirne
	un’altra, nuova: e su una tale base hanno decretato, come cosa certa, che le risoluzioni degli Dei superano di gran lunga il comprendonio umano.} Questo trovato, da solo, sarebbe
	stato sufficiente a che la verità restasse in eterno nascosta al genere umano, se la Matematica – che si occupa non dei fini, ma delle essenze e delle proprietà delle figure - non avesse
	mostrato agli umani un altro criterio di verità; e oltre alla Matematica si può indicare, senza che sia necessario enumerarli qui, altri fattori, grazie ai quali ha potuto accadere che taluni umani si siano accorti della natura di pregiudizio che hanno queste credenze comuni e
	siano riusciti a giungere alla vera conoscenza delle cose.
	[\dots omissis \dots]\\
	Aggiungerò tuttavia ancora un’osservazione: che questa dottrina dei fini sconvolge
	completamente la natura. Essa infatti considera come effetto ciò che invero è causa, e viceversa; poi mette dopo ciò che per natura è prima; e infine riduce imperfettissimo ciò che
	per natura è supremo e perfettissimo. Lasciamo da parte i primi due punti, che sono evidenti di per sé. Quanto al terzo, come risulta dalle Proposizioni 21, 22, 23, è perfettissimo
	quell’effetto che è prodotto da Dio immediatamente, ed una cosa è tanto più imperfetta
	quante più sono le cause intermedie di cui essa ha bisogno per essere prodotta: ma se le cose che sono state prodotte immediatamente da Dio fossero state fatte perché Dio conseguisse un suo fine, allora le ultime - a causa delle quali sono state fatte le prime - sarebbero
	necessariamente le più eccellenti. Inoltre, questa dottrina annienta la perfezione di Dio: dato che necessariamente, se agisce in vista di un fine, Dio manca di qualcosa, che desidera e
	cerca. E quantunque i teologi e i metafisici distinguano tra fine di indigenza (Dio creerebbe
	le cose perché ne ha bisogno) e fine di assimilazione (Dio vuole le cose siano per attribuire
	ad esse la sua beatitudine), essi tuttavia confessano che Dio ha fatto tutte le cose per se
	stesso, non per le creature: essi infatti non possono trovare che prima della creazione ci
	fosse un qualche Ente, oltre a Dio, a causa del quale Dio operasse; e pertanto debbono necessariamente ammettere che Dio mancava delle cose di cui ha predisposto l’esistenza, e le
	desiderava: come è evidente da sé. \textbf{Non si deve poi passar sotto silenzio che i seguaci di
	questa dottrina, i quali con l’individuare i fini delle cose hanno voluto mettere in mostra il
	loro ingegno, hanno - per rendere plausibili le loro affermazioni - escogitato una nuova maniera di argomentare: la riduzione non all’assurdo, ma all’ignoranza: e questo mostra
	che per sostenere la loro dottrina non c’era alcun argomento vero.} Per fare un esempio, se
	una tegola è caduta da un tetto sulla testa di qualcuno e l’ha ucciso, essi dimostrano nel
	modo seguente che la tegola è caduta per uccidere quell’uomo. Se la tegola non è caduta
	per volontà di Dio al fine predetto, chiederanno, come mai tante circostanze (perché spesso
	sono molte a concorrere) hanno potuto concorrere casualmente? Qualcuno risponderà che
	il caso avvenne perché tirava vento e perché l’uomo aveva bisogno di passare di là. Ed essi
	diranno: e perché il vento soffiò proprio allora? E perché quell’uomo doveva passare di là
	proprio nello stesso tempo? Qualcuno replicherà che il vento s’era levato proprio allora
	perché il giorno precedente, mentre il tempo era ancora calmo, il mare aveva cominciato
	ad agitarsi; e l’uomo era stato invitato da un amico. Ed essi di nuovo - perché si può domandare all’infinito: perché il mare era mosso? perché l’uomo era stato invitato in quel
	momento? E non smetteranno di chiedere le cause delle cause fin che \textbf{l’interlocutore non si
	rifugerà nella \textsc{volontà di Dio, cioè nel ricovero dell’ignoranza.}} Per fare un altro esempio, i
	seguaci della dottrina dei fini stupiscono quando si pongono a considerare la struttura del
	corpo umano: e, siccome ignorano le cause di un così mirabile meccanismo, concludono
	che esso non s’è costruito da sé per certe sue leggi intrinseche, ma è il prodotto di un’arte
	divina o soprannaturale, dalla quale esso è stato congegnato in maniera che un pezzo non
	danneggi l’altro, o, piuttosto, che ogni pezzo cooperi con ogni altro. \textbf{Vigendo tali criteri accade che chi vuol conoscere le vere cause degli eventi miracolosi, come chi cerca di capire
	da scienziato le cose della natura e non di meravigliarsene da sciocco, sia in generale giudicato eretico ed empio e proclamato tale da coloro che il volgo venera come interpreti della
	natura e degli Dei. Costoro sanno infatti che eliminando l’ignoranza si distrugge anche lo
	stupore, cioè l’unico mezzo che essi hanno di conservare credibile e di salvaguardare la loro
	autorità.} \\
	Ma ora lascio questo argomento per passare a quello che ho stabilito di trattare in
	terzo luogo.
	Essendosi persuasi che tutto ciò che accade è finalizzato a loro, gli umani hanno dovuto
	arrivar a giudicare che in ogni cosa il più importante è ciò che è più utile a loro, e che le cose più eccellenti sono quelle che danno a loro maggior piacere. Su questa base essi hanno,
	logicamente, dovuto formare le nozioni con le quali potere spiegare la natura delle cose:
	cioè le nozioni di Bene, di Male, di Ordine, di Confusione, di Caldo, di Freddo, di Bellezza,
	di Bruttezza; e dalla convinzione di esser liberi, che essi hanno, sono poi sorte le nozioni di
	Lode e di Biasimo, di Peccato e di Merito. Di queste ultime nozioni mi occuperò più avanti,
	dopo avere trattato della natura umana; qui invece spiegherò brevemente le prime. Gli
	umani dunque hanno chiamato Bene tutto ciò che favorisce la salute e inclina al culto di
	Dio, e Male ciò che è contrario a queste cose. Essi, poiché non penetrano intellettualmente
	la natura delle cose, ma si limitano all’apparenza di esse, che colpisce la loro immaginazione, non possono - prendendo l’immaginazione per l’intelletto - esprimere sulle cose giudizi
	corrispondenti al vero; e così, ignari della natura delle cose, e anche della natura propria,
	credono fermamente che nelle cose ci sia un ordine. Infatti, quando determinate cose sono
	disposte in maniera tale che noi, dopo averle considerate, possiamo facilmente figurarcele
	nella mente e quindi facilmente ricordarle, noi le diciamo bene ordinate; quando invece accade il contrario noi diciamo quelle cose male ordinate o confuse. E poiché le cose che noi
	immaginiamo facilmente ci piacciono più delle altre, gli umani preferiscono l’ordine alla
	confusione - come se l’ordine della natura fosse non qualcosa che vi scopre la nostra immaginazione, ma una realtà: e dicono che Dio ha creato le cose con ordine, attribuendo con
	ciò a Dio, senza saperlo, un’immaginazione, o magari convincendosi che Dio, a favore
	dell’immaginazione umana, abbia disposto le cose in modo da poter essere immaginate con
	la maggior agevolezza; e forse, avviati gli umani su questa strada, non li tratterrà il riflettere che ci sono infinite cose che superano di parecchio la nostra immaginazione, e molte che
	la confondono, debole com’è.\\
	Ma di questo ho detto abbastanza. Per quanto concerne le altre nozioni, anche esse non sono altro che modi dell’immaginare, dai quali l’immaginazione
	è variamente interessata: ma gl’ignoranti le considerano attributi principali delle cose, dato
	che, come abbiamo già detto, essi credono che tutte le cose siano state prodotte in vista di
	loro stessi, e chiamano le cose buone o cattive, sane o guaste o marce, a seconda del modo
	in cui ne sono toccati. Per esempio, se la sollecitazione che arriva ai nervi dagli oggetti percepiti attraverso gli occhi procura un senso di benessere, gli oggetti che ne sono causa sono
	chiamati belli; gli oggetti da cui proviene una sollecitazione sgradevole sono chiamati brutti. Gli oggetti poi che sollecitano i nervi tramite l’odorato sono, a loro volta, detti profumati
	o maleodoranti; quelli che sono percepiti dalla lingua sono detti dolci o amari, saporiti o insipidi; quelli che sono percepiti dal tatto sono detti duri o molli, ruvidi o lisci; di quelli, infine, che sollecitano i nervi per il tramite degli orecchi, si dice che producono un rumore, o
	un suono, o un’armonia. A proposito di quest’ultimo caso la follia degli umani è arrivata al
	punto di credere che dell’armonia si diletti anche Dio; e nemmeno mancano filosofi profondamente convinti che i movimenti dei corpi celesti producano un’armonia. Tutti questi fatti mostrano a sufficienza che sulle cose ciascuno ha espresso giudizi conformi alle caratteristiche del suo cervello, o, meglio, che la gente ha preso, in luogo delle cose, ciò che la sua
	immaginazione risentiva delle cose stesse. Per questo motivo non c’è da meravigliarsi (notiamo di passaggio anche questo) che tra gli umani siano sorte tutte le controversie filosofiche che conosciamo così bene, e che da esse sia infine uscito lo Scetticismo. Le strutture dei diversi corpi umani sono simili in molti aspetti, ma sono dissimili in moltissimi altri: e per questo ciò che a uno pare buono, a un altro pare cattivo; quel che per uno è
	ordinato, per un altro è confuso; quel che a uno fa piacere, a un altro fa dispiacere. Potrei
	continuare, ma mi fermo qui, sia perché non è questa la sede per diffondersi su un tale argomento, sia perché tutti ne hanno fatto sufficiente esperienza: tutti infatti sanno che
	quante teste, tanti pareri; che ognuno stima d’aver giudizio anche più del necessario; che
	ci son tante differenze fra le idee quante fra i gusti: detti, questi, che mostrano a sufficienza come gli umani giudichino delle cose secondo la disposizione del loro cervello, e come le
	immaginino più che comprenderle. Se infatti gli umani le comprendessero mediante l’intelletto, le cose nella loro realtà - come testimonia la Matematica - potrebbero magari non
	attrarre tutti, ma almeno convincere tutti alla stessa maniera.
	È dunque evidente che tutte le nozioni con le quali la gente è usa a "spiegare" la natura
	non sono altro che modi dell’immaginazione, e non chiariscono la struttura interna di alcunché ma soltanto ci informano sulla costituzione dell’immaginazione; e poiché questi enti hanno dei nomi, come se si trattasse di realtà esistenti fuori dell’immaginazione, io li chiamo enti non di ragione, ma d’immaginazione; e così è facile confutare tutti gli argomenti che vengono tratti da quelle nozioni contro il nostro modo di vedere. Molti infatti sogliono argomentare così: se tutte le cose sono uscite dalla necessità della perfettissima natura di Dio, di dove provengono dunque alla natura tante imperfezioni: le cose che si guastano fino a puzzare, le cose tanto brutte da suscitare la nausea, il disordine, il male, il peccato, eccetera? Ma, l’ho detto or ora, è facile confutare quei tali. La perfezione delle cose,
	infatti, si deve valutare solo in riguardo della loro natura e della loro potenza (di agire, cioè la capacità  di causare altri affetti a loro volta); e le cose non
	sono più o meno perfette a seconda che dilettano o urtano i sensi degli umani, a seconda
	che sono gradite alla natura umana o ad essa ripugnano.\footnote{Etica prima parte appendice.}
	
\end{quotation}
\newpage
\section{Seconda parte - La mente umana}
Chiarito che il mondo è una derivazione geometrica dell'unica sostanza (la Natura), che quindi procede secondo leggi e regole interne necessarie, ovviamente senza scopi e arbitrio, Spinoza procede a collocare l'umanità nell'universo: essa non è una particolarità o una diversità che spicca rispetto al resto del mondo, bensì è uno dei tanti modi della Natura, per cui va studiato e capito secondo leggi fisiche e naturali.

In Spinoza vi è un grandissimo contrasto con le visioni religiose che mettono l'uomo al centro del mondo creato: per esempio anche il quasi contemporaneo Cartesio (cui Spinoza deve l'approccio razionalistico), vedeva l'essere umano come l'unico dotato di \textit{res cogitans} (pensiero), e, il resto del mondo, come \textit{res extenza}, cioè sostanze che obbediscono a leggi meccaniche, quindi oggetti inanimati o degli automi nel caso degli animali, di cui disporre a proprio piacimento.

Il passo successivo di Spinoza è spiegare cosa è l'uomo, in particolare la sua mente.

\begin{quotation}
	\small I modi di qualsiasi attributo hanno Dio come causa solo in quanto egli è considerato sotto l’attributo per mezzo del quale i modi in esame sono concepiti, e non in quanto egli sia considerato sotto qualsiasi altro attributo.\footnote{Etica seconda parte proposizione 6}
	
	\small L’ordine e la connessione delle idee sono identici all’ordine e alla connessione delle cose.\footnote{Etica seconda parte proposizione 7}
	
	\small La Mente e il Corpo stesso costituiscono un unico Individuo,
	che è concepito ora come modo dell’attributo "Pensiero", ora come modo dell’attributo "Estensione".\footnote{Etica seconda parte scolio alla proposizione 21}
	
\end{quotation}
Vi è un unico ordine del pensiero e dell'estensione, cioè delle menti e dei corpi, ed entrambi hanno uguale dignità perché hanno la Natura come unica origine, ora vista sotto un attributo, ora sotto un altro. Questa è la dottrina del parallelismo di Spinoza, in cui la serie del corpo e la serie della mente presentano il medesimo ordine e concatenamento secondo stessi principi. Quando muoviamo un oggetto, nella nostra mente contemporaneamente abbiamo l'idea del movimento che stiamo compiendo con il corpo; così quando un oggetto ci colpisce.

\begin{quotation}
	\small L’oggetto dell’idea che costituisce la Mente umana è il Corpo, ossia un determinato modo, esistente in atto (cioè effettivamente e presentemente), dell’Estensione, e nient’altro.\footnote{Etica seconda parte proposizione 13}
\end{quotation}

Il corpo e la mente sono uniti, e come il corpo è formato da numerose parti, tale la mente lo è da moltissime idee.

\begin{quotation}
	\small L’idea di qualsiasi maniera in cui il Corpo umano è \textbf{affetto} da corpi esterni, deve implicare la natura del Corpo umano e insieme la natura del corpo esterno.\footnote{Etica seconda parte proposizione 16}
\end{quotation}

Per \textbf{affezioni} si intendono la concretizzazione della sostanza (i modi), o la sua variazione, \textbf{le modificazioni del modo}. Queste affezioni sono perciò tracce corporee, di cui la nostra mente ricava idee: queste sono l'unica via della mente per conoscere il corpo; in particolare le idee dei corpi esterni indicano più la natura del nostro corpo (e i cambiamenti che subisce) che dei corpi esterni.\\
Ogni superiorità dell'anima sul corpo è rifiutata, la morale viene completamente capovolta, non c'è nessuna impresa di dominio delle passioni da parte della coscienza, bensì ciò che è azione nell’anima è anche necessariamente azione nel corpo, ciò che è passione nel corpo è anche necessariamente passione nell’anima. Nell'affermazione di Spinoza: "Nessuno sa ciò che può il corpo…\footnote{Etica seconda parte scolio della proposizione 2}", Gilles Deleuze vi ha visto una scoperta dell'inconscio:

\begin{quotation}
	\small Si tratta di mostrare che il corpo va oltre la conoscenza che se ne ha, e che nondimeno il pensiero oltrepassa la coscienza che se ne ha. Non vi sono meno cose nella mente che oltrepassano la nostra coscienza che cose nel corpo che sorpassano la nostra conoscenza. È dunque per un solo e medesimo movimento che arriveremo ad afferrare la potenza del corpo al di là delle condizioni date della nostra conoscenza e a cogliere la potenza della mente al di là delle condizioni date della nostra coscienza. Si cerca di acquisire una conoscenza delle potenze del corpo per scoprire parallelamente le capacità della mente che sfuggono alla coscienza, per poter comparare le potenze. In breve, il corpo, secondo Spinoza, non implica alcuna svalorizzazione del pensiero in rapporto all’estensione, ma, cosa assai più importante, una svalorizzazione della coscienza in rapporto al pensiero, una scoperta dell’inconscio, e di un inconscio del pensiero, non meno profondo che l’ignoto del corpo.\footnote{Gilles Deleuze, Spinoza Filosofia pratica.}
\end{quotation}

Quindi le idee che si formano nella nostra mente dovute alle affezioni sono \textbf{inadeguate}, rappresentano ciò che capita al nostro corpo, le tracce di un corpo esterno sul nostro, una mescolanza di due corpi. Inadeguate sono pure le idee delle idee di affezioni, cioè le immagini e i ricordi che formiamo nella nostra mente, che spesso si legano tra di loro in base all'ordine cui ci sono apparse, pur senza un vero rapporto causa effetto: le astrazioni (o gli universali se si usa un termine caro alla Scolastica) generate da queste idee a loro volta saranno inadeguate.

Dunque, quali sono le idee \textbf{adeguate}? Sono quelle riferite a Dio, cioè di cui conosciamo le leggi che le hanno determinate, i rapporti causa effetto. Da idee adeguate, possiamo ricavarne altre adeguate, ovvero le dimostrazioni logiche portano a nuova conoscenza.\\L'idea inadeguata è come una conseguenza senza le proprie premesse, sfugge alla nostra comprensione attenendosi ad un ordine di incontri fortuiti anziché raggiungere la concatenazione delle idee: quindi il falso non è positivo, semplicemente è carenza di conoscenza, e un'idea inadeguata può comunque costituire un punto di partenza per arrivare a quella adeguata.\\
L'idea adeguata sarà ricavata grazie alle \textbf{nozioni comuni}: queste non sono concetti universali e astratti, bensì idee più o meno generali che rappresentano la concordanza di due o più corpi; riferendo le nozioni comuni all'uomo, esse sono la comprensione dei rapporti che entrano nella composizione tra l'uomo e un altro corpo (con cui si interagisce), quando questa composizione porta ad un accrescimento del nostro essere (diremo meglio successivamente riguardo l'istinto ad incrementare la propria essenza). Per chiarire, quando un corpo interagisce con il nostro, entrambi subiscono una modificazione, da cui deriva conoscenza inadeguata. Se tra i due corpi vi è qualcosa in comune, questa parte non si modificherà, e potrà essere compresa in modo adeguato mediante l'uso della ragione, che andrà a scovare le regole universali di questa congruenza che concatena tutte le cose tra loro (tutti i modi sono dotati degli stessi attributi, quindi qualcosa in comune vi è sempre): regole eternamente valide, \textit{sub specie aeternitatis}.

Per Spinoza, cercare cose più possibili simili a noi è importante, perché, come dimostra, più il corpo umano e la mente interagiscono tra di loro per cogliere le affezioni esterne al corpo, più la mente apprende, se vengono colte le nozioni comuni che ci conducono ad idee adeguate. Infatti afferma:

\begin{quotation}
	\small [...] come un Corpo è più idoneo di altri a fare nello stesso tempo diverse cose o a riceverne l’azione, così proporzionalmente la sua Mente è più idonea di altre a ricevere nello stesso tempo diverse informazioni; e quanto più le azioni di un determinato Corpo dipendono da questo Corpo
	solo, e quanti meno altri corpi concorrono al suo agire, con tanto maggiore chiarezza la
	Mente corrispondente è idonea a comprendere. Grazie a queste considerazioni possiamo
	conoscere come una mente eccella sulle altre.\footnote{Etica seconda parte scolio alla proposizione 13.}
\end{quotation}

Apprendere da corpi che hanno nozioni comuni con noi porta ad accrescere la nostra conoscenza; come chiariremo successivamente, interagendo con cose simili accresciamo la nostra essenza, perciò saremo spinti a fare esperienze positive con i modi che più ci sono simili, cioè altri esseri umani.

Stabilito in che modo la mente impara, Spinoza classifica i gradi conoscenza possibili:
\begin{enumerate}
	\item Conoscenza del primo genere, o opinione, o immaginazione: la conoscenza che deriva dalla percezione sensibile della singola cosa, che dai sensi ci viene proposta all’intelletto in maniera disordinata o casuale; e inoltre la conoscenza che segue dai "segni" (il linguaggio scritto o parlato), che porta a immaginare le cose. Questo primo genere esprime la condizione naturale della nostra vita fino a che abbiamo idee inadeguate, dove vediamo solo gli effetti del mondo su di noi, senza comprendere le ragioni per cui tutto sembra dovuto al caso, compreso il nostro patire (subire passivamente l'affezione, sia positiva che negativa).
	\item Conoscenza del secondo genere o ragione: deriva dal nostro avere nozioni comuni e idee adeguate delle proprietà delle cose, cioè la composizione dei legami, lo sforzo della ragione di organizzare gli incontri fra corpi (o idee, visto il parallelismo) esistenti secondo rapporti che si "compongono" (che accrescono l'essenza di una cosa). Questa conoscenza e quella successiva del terzo genere ci insegnano a distinguere il vero dal falso.
	\item Conoscenza del terzo genere o sapere intuitivo: ovvero quella colta per intuito e che riguarda la singola essenza del modo (idea o corpo). Qui Spinoza appare un po' mistico, io interpreto questa conoscenza come l'intuizione che ci fa cogliere immediatamente la verità quando siamo molto pratici di ragionamenti deduttivi e dimostrativi; in ogni caso nella parte quinta dell'Etica questo tema viene rispeso.
\end{enumerate}

Ai generi di conoscenza corrispondono tre modi di esistenza, poiché  il conoscere influenza la coscienza che abbiamo di quanto ci circonda:

\begin{quotation}
	\small È proprio della natura della Ragione considerare le cose non come contingenti ma come
	necessarie. [...]\\
 Corollario 1: di qui deriva che il nostro considerare le cose come contingenti, tanto
	rispetto al passato quanto rispetto al futuro, dipende solo dall’immaginazione. [...]\\Corollario 2: è proprio della natura della Ragione percepire le cose nella loro peculiare eternità (\textit{sub specie aeternitatis}), ossia considerare gli aspetti anche transitori della Sostanza come partecipi,
	in un modo loro peculiare, dell’essere eterno della Sostanza stessa.\footnote{Etica seconda parte proposizione 44.}
\end{quotation}

Ovvero ogni idea adeguata implica la conoscenza eterna e infinita di Dio, cioè di quelle leggi di natura che governano il mondo. Questa visione era presente anche in Galileo Galilei: possiamo conoscere con la stessa intensità di Dio, non con la stessa estensione: alcune cose, le leggi della fisica, possono essere apprese dall'uomo con lo stesso grado di comprensione di Dio. Questa visione del mondo costò a Galileo il ben noto processo.

Chiarito che la conoscenza vera è quella che svela tutti i necessari rapporti di natura, Spinoza rimarca l'infondatezza del concetto di volontà:

\begin{quotation}
	\small Nella Mente non c’è alcuna volontà indipendente o libera: ma nel volere questa cosa o
	quella la Mente è determinata da una causa, che è determinata anch’essa da un’altra causa,
	la quale a sua volta è determinata da un’altra, e così in infinito.\footnote{Etica seconda parte proposizione 48}
	
	\small Nella Mente non c’è alcuna volizione, cioè non c’è alcuna affermazione o negazione, oltre
	a quella che un’idea, in quanto è idea, implica.\\Corollario: la volontà e l’intelletto sono la stessa e unica cosa.\footnote{Etica seconda parte proposizione 49}
\end{quotation}

Questa parte dell'Etica viene conclusa da un lungo scolio, di cui riporto alcuni punti, che ci fanno capire come la filosofia possa farci stare meglio:

\begin{quotation}
	\small Questa dottrina ci insegna infatti che noi operiamo grazie soltanto al volere di Dio e
	che siamo partecipi della natura divina, e questo tanto più quanto più perfette sono le azioni che compiamo e quanto più e più profonda è la nostra conoscenza di Dio. ...\\
	Questa dottrina c’insegna come dobbiamo comportarci riguardo alle cose fortuite ossia estranee al nostro potere, cioè riguardo alle cose che non dipendono dalla nostra natura
	e dalle sue facoltà: appunto, aspettare e vivere senza alcun patema d’animo le manifestazioni della "fortuna" e della "sfortuna": cosa del tutto ragionevole, poiché tutti gli eventi procedono dall’eterna determinazione di Dio con la stessa necessità con cui dalla natura del
	triangolo procede che la somma dei suoi tre angoli interni equivalga a due angoli retti. ...\\
	Questa dottrina giova alle relazioni sociali in genere in quanto insegna a non odiare né disprezzare né deridere alcuno, e a non adirarsi con alcuno, e a non invidiare alcuno; e
	inoltre in quanto insegna che ognuno sia contento del suo, e sia d’aiuto al prossimo non
	per una pietà sentimentale o per parzialità o per superstizione, ma soltanto in conformità
	di quel che suggerisce la Ragione secondo le esigenze del tempo e dei casi: come mostrerò
	nella Quarta Parte.\\
	Questa dottrina, infine, giova non poco alla collettività organizzata o comunità politica, in quanto insegna con quale criterio i cittadini debbano essere governati e diretti: appunto non perché agiscano da schiavi, ma perché scelgano liberamente di compiere ciò che
	è il meglio.\footnote{Etica seconda parte scolio alla proposizione 49.}
\end{quotation}
\newpage
\section[Gli affetti]{Terza parte - Origine e natura degli affetti}

Dopo aver trattato della mente, Spinoza passa ai sentimenti:  quelli negativi non vanno ripudiati e respinti, come sempre è stato fatto dalla morale religiosa e filosofica, bensì essi

\begin{quotation}
	\small procedono dalla stessa necessità e dalla stessa virtù della natura da cui divengono tutte le altre cose singole; e quindi riconoscono cause determinate, mediante le quali
	essi sono compresi, ed hanno determinate proprietà, degne d’esser conosciute da noi esattamente come le proprietà di qualsiasi altra cosa di quelle della cui contemplazione ci dilettiamo. Con lo stesso metodo, pertanto, col quale nelle pagine precedenti ho trattato di Dio
	e della Mente, tratterò ora della natura e delle forze dei Sentimenti, e del potere che la
	Mente ha su di essi; e considererò le azioni e le inclinazioni umane come se fosse questione
	di linee, di superfici e di solidi.\footnote{Etica terza parte prefazione.}
\end{quotation}

Ovvero desideri ed emozioni sono frutto della meccanica dei corpi, non vanno giudicati moralmente così come non valutiamo la moralità di un sasso; gli impulsi vanno invece compresi per capire come influenzino la consapevolezza, il comportamento e le azioni.

\begin{quotation}
	\begin{center}
	\textbf{Definizioni}	
	\end{center}

\small Chiamo causa adeguata quella del cui effetto si può avere percezione e conoscenza
	chiare e distinte per mezzo di essa; chiamo invece causa inadeguata o parziale quella il cui
	effetto non può essere inteso per mezzo di essa sola.
	
	Dico che noi agiamo, o siamo attivi, quando in noi o fuori di noi accade qualcosa di
	cui noi siamo la causa adeguata: cioè quando dalla nostra natura deriva, in noi o fuori di noi, qualcosa che può essere inteso in maniera chiara e distinta per
	mezzo unicamente di tale nostra natura. Viceversa, dico che noi patiamo, o siamo passivi,
	quando in noi accade qualcosa, o dalla nostra natura segue qualcosa, di cui noi non siamo
	causa se non in parte.
	
	 Posto che le affezioni del nostro Corpo sono le reazioni del Corpo stesso agli enti e agli
	eventi dai quali il Corpo è interessato o dei quali risente: affezioni dalle quali la capacità (o potenza) di
	agire del Corpo stesso è aumentata o diminuita, favorita od ostacolata; intendo per affetti le affezioni qui descritte e, insieme, le idee di queste affezioni.
	Nel caso, quindi, in cui noi possiamo esser causa adeguata di qualcuna di queste affezioni, per affetto intendo un nostro essere attivi, cioè un’azione; altrimenti intendo un
	nostro essere passivi, cioè una passione.
	
	\begin{center}
	\normalsize \textbf{Postulati}	
	\end{center}

	\small Il Corpo umano può essere interessato da vari fattori in molte maniere, dalle quali la
	sua potenza o capacità di agire è aumentata o diminuita, e anche in altre maniere che non
	rendono maggiore né minore la sua potenza o capacità predetta.
	
	Il Corpo umano può subire molti cambiamenti, e nondimeno conservare le impressioni o tracce degli oggetti, e di conseguenza le immagini stesse delle cose.\footnote{Prime tre definizioni e primi due postulati parte tre dell'etica.}
\end{quotation}

Agiamo quando conosciamo la causa (adeguata) che ci muove all'azione, altrimenti patiamo: da notare che quando agiamo senza conoscere (patiamo in termini spinoziani), pensiamo di scegliere e quindi di essere attivi, mentre in verità non sappiamo le ragioni del nostro impulso.

Per essere precisi e usare una terminologia più moderna che ci verrà utile nel seguito, avremo\footnote{mi rifaccio ai termini usati in: A. Damasio, Alla ricerca di Spinoza, Adelphi, Milano, 2003.}:

\begin{itemize}
	\item affetti o emozioni (termine più al passo coi tempi) indicano i cambiamenti della potenza di agire del nostro corpo da parte di corpi esterni (le affezioni).
	\item sentimenti indicano le idee di questi affetti: ogni emozione produce un sentimento. Il sentimento è l’idea del corpo,  è l’immagine mentale di uno stato particolare corporeo, è l’idea di un suo
	stato emotivo; il sentimento è l’idea di una o più emozioni ovvero di affezioni corporee.
	\item passione è dovuta ad un affetto di cui non conosciamo la causa, di cui abbiamo idea inadeguata.
\end{itemize}

Nelle successive proposizioni, Spinoza spiega la causa che spinge ogni essere a patire che ad agire:

\begin{quotation}
	\small Ciascuna cosa, per quanto sta in essa (ossia per quanto essa può), si sforza di perseverare
	nel suo essere.\footnote{Etica terza parte proposizione 6.}
	
	Lo sforzo (\textit{conatus}) con cui ciascuna cosa tende a perseverare nel suo essere non è altro che l’essenza attuale della cosa stessa, cioè il suo essere, e il suo esserci, presente ed attivo.\footnote{Etica terza parte proposizione 7.}
	
	La Mente, sia in quanto ha idee chiare e distinte, sia in quanto ha idee confuse, si sforza
	di perseverare nel suo essere per una durata indefinita, ed è \textbf{consapevole} di questo suo sforzo.\footnote{Etica terza parte proposizione 9.}
\end{quotation}

Siamo mossi dal \textit{conatus}, lo sforzo con cui ogni cosa cerca di perseverare nel suo essere (istinto di sopravvivenza), di migliorare la propria condizione, di realizzare al massimo grado la propria essenza. Importante notare che di questa spinta siamo sempre consapevoli, per cui pensiamo di volere una certa cosa qualora non conosciamo le cause di questo istinto naturale.\\
Questo perseverare nel proprio essere di ogni modo finito è riflesso dell'infinita potenza divina, è una sorta di principio di inerzia applicato alla vita tangibile: ogni modo persevera nel suo stato a meno che non sia costretto a mutare quello stato da forze impresse.\footnote{Il principio di inerzia recita: ogni corpo persevera nel suo stato di quiete o di moto uniforme e rettilineo a meno che non sia costretto a mutare quello stato da forze impresse.}\\
Il \textit{conatus} è potenza di agire, cioè la tendenza a mantenere e accrescere l'attitudine di venire affetti: questa è ad ogni istante soddisfatta dapprima dalle affezioni, che generano affetti (emozioni), prodotti da altri modi esistenti (altri corpi); queste emozioni generate dall'esterno sono immagini, ricordi e passioni che il \textit{conatus} prende quando è determinato a fare questo o quello da un'affezione che gli capita. Perciò le emozioni determinano il \textit{conatus} come causa della coscienza: il nostro "perseverare a vivere", divenuto cosciente di sé sotto questo o quell'affetto si chiama desiderio, essendo desiderio di qualche cosa.



Se il corpo esterno accresce il nostro essere, ci darà gioia (\textit{letizia}), se lo diminuisce, tristezza. Ribadendo in altri termini quanto sopra detto, la nostra coscienza non è altro che il sentimento del passaggio da gioa a tristezza o viceversa, a testimoniare le variazioni del nostro appetito in funzione degli altri corpi o delle altre idee.

Il \textit{conatus} è sempre in atto, costantemente accresciamo la nostra esistenza cercando sensazioni di gioia, è un diritto di natura: se ci affidiamo alla casualità degli incontri con altri modi, e all'arbitrio delle affezioni e degli affetti che determinano il \textit{conatus} stesso da di fuori, la nostra vita sarà solamente un provare passioni gioiose e al contempo distruggere le minacce; tali distruzioni portano rancore e sentimenti legati alla tristezza, oltre che il rischio di incontrare una cosa più potente di noi che ci annienterà.\\
Per questo lo sforzo di accrescere la nostra potenza deve indurre l'uomo ad ottenere sentimenti gioiosi tramite l'organizzazione di buoni incontri: cioè \textbf{incotrare i modi che convengono con la nostra natura e si compongono con noi. Questo sforzo è quello della ragione, che ci fa entrare in possesso della potenza di agire, cioè della capacità di provare gioie attive che derivano da idee adeguate che la ragione ci ha fornito (conoscenza). Questo \textit{conatus} inteso come sforzo ragionato a perseverare nella propria esistenza, è la virtù.}\footnote{questa parte che distingue azioni e passioni, agire secondo ragione o patire è più argomento della quarta parte dell'etica, ma mi è sembrato giusto introdurlo già adesso.} I modi che si compongono con la nostra essenza, accrescendola, sono quelli con cui abbiamo più "nozioni comuni", che accrescono anche la conoscenza, come già detto nella parte due dell'Etica: la dimostrazione dell'acrescimento di potenza tra modi simili o uguali sarà oggetto nella quarta parte.

Nello scolio all'ultima proposizione citata Spinoza va oltre la morale, si muove "al di là del bene e del male", citando Friedrich Nietzsche.

\begin{quotation}
	\small Questo sforzo, quando si riferisce alla sola Mente, si chiama Volontà; ma
	quando si riferisce insieme alla Mente e al Corpo si chiama Appetito: il quale perciò non è
	altro che l’essenza stessa dell’Uomo, dalla natura del quale deriva necessariamente ciò che
	è indirizzato alla sua conservazione: precisamente ciò, quindi, che l’Uomo è determinato
	ad operare. Fra l’Appetito e la Cupidità (desiderio) non c’è poi alcuna differenza, almeno per quanto
	concerne gli umani, ai quali perlopiù si attribuisce la Cupidità: essi infatti sono consci del loro Appetito; e pertanto la Cupidità può appunto definirsi così, un Appetito che si ha la
	coscienza d’avere. \textbf{Da tutte queste considerazioni risulta dunque che noi non ci rivolgiamo
		con interesse verso una qualche cosa} - né la vogliamo, o la desideriamo istintivamente, o la
	desideriamo consapevolmente - \textbf{perché giudichiamo che essa sia buona; ma, al contrario,
		noi giudichiamo buona una cosa perché essa risveglia il nostro interesse, o perché la vogliamo, o perché la desideriamo, istintivamente o consapevolmente.}
\end{quotation}

\textbf{Non desideriamo il bene; bene è ciò che desideriamo, e corrisponde a quello che è utile per accrescere il nostro essere e appagare il nostro desiderio di vita.}

Inoltre i sentimenti sono causati dalle immagini delle cose e dai ricordi, che spesso si concatenano tra loro.\\
Dagli affetti primari di gioia, tristezza e desiderio, Spinoza ricava le cause e spiega tutti i sentimenti e fluttuazioni dell'animo umano. Ne riportiamo alcuni, rimandando alla lettura dell'etica per gli altri:

\begin{quotation}
	\small L’Amore non è appunto altro che Gioia accompagnata dall’idea di una causa
	esterna; e l’Odio non è altro che Tristezza accompagnata dall’idea di una causa esterna.
	Da quanto precede vediamo inoltre che chi ama si sforza necessariamente di aver presente
	e di conservare la cosa che egli ama, mentre al contrario chi odia si sforza di allontanare e
	di distruggere la cosa che egli ha in odio.\footnote{Etica terza parte scolio alla proposizione 13.}
	
	La Speranza non è altro che una Gioa instabile originata dall’immagine di una cosa futura (o anche passata) del cui esito dubitiamo. Il Timore, al contrario, è una Tristezza,
	anch’essa instabile, originata dall’immagine di una cosa dall’esito dubbio. Se da questi due
	sentimenti si toglie il fattore dubbio se ne ottiene rispettivamente la Sicurezza e l’Angoscia
	senza rimedio, ossia una Gioa, o una Tristezza, originata dall’immagine della cosa che
	abbiamo sperato o temuto. L'Esultanza poi è una Gioa nata dall’immagine di una cosa passata, del cui esito abbiamo dubitato. Il Rincrescimento, infine, è la Tristezza opposta all'Esultanza.\footnote{Etica terza parte scolio 2 alla proposizione 18.}
	
	La Superbia è dunque una Letizia
	che sorge da questo, che un umano valuta se stesso più del giusto.\footnote{Etica terza parte scolio alla proposizione 26.}
\end{quotation}

Interessanti alcuni altri passaggi nella trattazione delle emozioni:

\begin{quotation}
	\small Vediamo quindi che per natura gli umani sono perlopiù congegnati in modo da aver compassione di chi deve sopportare un male, e da aver invidia di chi può godersi
	un bene, e ciò con una malevolenza tanto maggiore quanto maggiore è
	l’amore per la cosa che essi immaginano posseduta da un altro. Vediamo inoltre che dalla
	stessa proprietà della natura umana dalla quale deriva che gli umani sono compassionevoli
	deriva anche il loro essere invidiosi e ambiziosi.\footnote{Etica terza parte scolio alla proposizione 32.}
	
 Per quel che se ne vede, gli umani
	sono parecchio più disposti a vendicarsi che a contraccambiare un beneficio.\footnote{Etica terza parte scolio alla proposizione 41.}
	
	L’Odio è accresciuto dall’Odio reciproco, e può, viceversa, essere annullato dall’Amore.\footnote{Etica terza parte proposizione 43.}
	
	L’Odio che è interamente vinto dall’Amore diventa esso stesso Amore; e l’Amore così originato è maggiore che se in precedenza non fosse stato Odio.\footnote{Etica terza parte proposizione 44.}
	
	L’Amore e l’Odio verso una cosa che immaginiamo libera debbono essere, a parità di motivi, maggiori che verso una cosa necessaria.\footnote{Etica terza parte proposizione 49.}
\end{quotation}

L'ultima proposizione citata afferma che, comprendendo le ragioni degli eventi (cause adeguate), le passioni di amore e odio si attenuano, ma soprattutto implica che \textbf{gli umani, poiché si credono liberi, si amano e si odiano vicendevolmente con un impegno maggiore di quello con cui amano o odiano gli altri esseri (non ritenuti liberi, come animali o oggetti inanimati)}.

A chiudere la terza parte dell'Etica:

\begin{quotation}
	\small Tutti i sentimenti che si riferiscono alla Mente in quanto essa è attiva hanno relazione
	esclusivamente con la Letizia e con la Cupidità (desiderio).\footnote{Etica terza parte proposizione 59.}
\end{quotation}

Questa affermazione esplica quanto detto in precedenza ed è un ponte verso la quarta. Perciò ribadiamo: quando agiamo, quindi sappiamo quello che facciamo (idee adeguate), accresciamo il nostro essere, ci procuriamo solo gioia; non avrebbe senso cercare tristezza consapevolmente. I cattivi sentimenti si presentano solo quando patiamo, quando si fanno cose senza avere ben chiara l'idea che ci ha spinto ad intraprendere un'azione. Si capisce così l'importanza della conoscenza, anche nella vita di tutti i giorni.

In tempi recenti molti studi neuroscientifici hanno confermato le intuizioni di Spinoza riguardo la genesi delle emozioni e dei sentimenti\footnote{A. Damasio, Alla ricerca di Spinoza, Adelphi, Milano, 2003, tutto quanto dirò fino alla fine del paragrafo proviene da questa fonte.}.

Il neuroscienziato Damasio riconosce le emozioni come momento precedente ai sentimenti, se pur non facciano parte di un processo separato ma costituenti un unico processo essendo
corpo e mente la stessa cosa. L'emozione precede
il sentimento (altrimenti quest'ultimo verrebbe a mancare), che possiamo vederlo come una mappa cerebrale in cui la mente immagina lo stato corporeo ovvero lo stato dell’ intero organismo. Il sentimento non è altro che l’idea del
corpo, l’immagine dell’emozione, è la coscienza del corpo. È sentire come stiamo in quanto unica entità e non sentire come sta il corpo, quasi fosse staccato dalla coscienza, dove la mente si limita a valutare la sua situazione. \\
Se il corpo sta male, la mente realizza di stare male; se il corpo prova
emozioni di benessere che lo inducono ad avere maggiore forza, la mente realizza di
essere felice. Siamo il nostro corpo. Ogni
sua sofferenza, ogni parte di esso, la mente li realizza sotto forma di sentimento. "I sentimenti […] non insorgono solo dalle emozioni vere e proprie, ma da qualsiasi insieme di
reazioni omeostatiche (cioè che cambiano l'equilibrio interno del corpo), e traducono nel linguaggio della mente lo stato vitale in cui versa
l’organismo".

Risulta completamente ribaltata la concezione dell'uomo basata sul dualismo anima-corpo, praticamente l'unica visione antropologica da Socrate (quinto secolo a.c.) a Cartesio (1596-1650, di poco precedente a Spinoza). La concezione duale di corpo e mente è insita ancora in noi, che seppur sforzandoci,
per quanto ci convinciamo, non riusciamo ad immaginare mente e corpo come un'unica
cosa. Lo stesso fatto di utilizzare due nomi separati e distinti, implica il pensare due
cose separate e distinte. Infatti, ancora oggi, molte teorie dividono il corpo dal cervello
ovvero dalla mente, come se il cervello non fosse esso stesso corpo. 

Le scoperte neurobiologiche hanno anche
evidenziato che i sentimenti di gioia fanno bene alla salute del corpo e al mantenimento del suo benessere ed
equilibrio, a differenza dei sentimenti tristi e dolorosi che indeboliscono la salute del
corpo; i sentimenti derivanti dalla gioia sono: "più favorevoli
alla salute e allo sviluppo creativo del nostro essere".
\newpage

\section[La schiavitù umana]{Quarta parte - La schiavitù umana, ossia le forze degli affetti}

Questa parte inizia con una prefazione degna di nota, in cui Spinoza si scaglia contro l'illusione della finalità delle azioni, la volontà e contro la creazione di modelli universali e trascendenti per stabilire gradi di perfezione che in natura non esistono; nel finale riafferma i concetti di potenza di agire, bene, male e utile. Sono pagine talmente belle, che non si possono riassumere, ma solo leggere.

\begin{quotation}
	\small Chiamo schiavitù l’impotenza degli umani a governare e a reprimere i sentimenti: dato che l’agire di un umano sottomesso ai sentimenti è guidato non dall’umano
	stesso, ma dalla sorte: in potere della quale egli si trova ad un punto tale, che spesso è costretto, sebbene veda ciò che per lui è meglio, a scegliere invece il peggio. Dimostrare la
	causa di questa situazione, e dimostrare inoltre che cosa i sentimenti abbiano di buono o di
	cattivo, è ciò che in questa Parte mi sono proposto. Ma prima di cominciare vorrei premettere poche parole a proposito della perfezione e dell’imperfezione e del bene e del male.
	Chi ha stabilito di fare una certa cosa, e l’ha portata a compimento, dirà che la sua opera
	è perfetta; e così dirà anche ognuno che conosca correttamente, o creda di conoscere, il
	pensiero e lo scopo dell’autore di quell’opera. Per esempio, se qualcuno vede un’opera - che
	suppongo non essere ancora compiuta - e sa che lo scopo dell’autore di quell’opera è, poniamo, la costruzione di una casa, dirà che la casa è incompiuta, o imperfetta; e la dirà invece compiuta, o perfetta, dal momento che l’avrà vista portata a quel compimento che
	l’autore aveva progettato di darvi. Ma chi veda un’opera della quale non abbia mai visto un
	altro esemplare, e non conosca il pensiero del costruttore, non potrà certo sapere se
	quell’opera sia perfetta o imperfetta. E sembra che questo sia stato il primitivo significato
	di tali termini. Ma dopo che gli umani han cominciato a formarsi idee universali, e a concepire modelli di case, di palazzi, di torri, eccetera, e a preferire determinati modelli di cose ad altri modelli, è accaduto che ognuno chiami perfetto ciò
	che gli sembri combaciare meglio con 1’ idea universale che egli s’è fatto di quella tal cosa, e
	imperfetto, al contrario, ciò che egli veda meno combaciante col modello da lui concepito -
	benché a giudizio dell’artefice dell’oggetto esso possa essere perfettamente compiuto. E
	non sembra che sia diversa la ragione dell’abitudine, che gli umani hanno, di chiamare perfette o imperfette anche le cose naturali, quelle cioè che non sono prodotte da mano umana: ché gli umani sogliono infatti formarsi, sia delle cose naturali sia delle cose artificiali,
	idee universali, che essi prendono come modelli delle cose, e che secondo loro la natura (la
	quale, secondo loro, non fa nulla senza un fine) tiene ben presenti e adotta anch’essa come
	modelli. Quando poi vedono che nella natura si presenta qualche cosa che s’adatta non
	completamente al loro modello ideale di quella cosa, essi credono allora che la natura stessa abbia avuto un mancamento o un ghiribizzo e abbia lasciato imperfetta la cosa considerata. Vediamo pertanto che gli umani si sono abituati a chiamare le cose "perfette" o "imperfette" più per pregiudizio che per una vera conoscenza delle cose stesse. Abbiamo infatti
	mostrato nell’Appendice della Prima Parte che la Natura non agisce in vista d’un fine: l’Ente eterno e infinito, che chiamiamo Dio, o Natura, opera per la medesima necessità per la
	quale esiste. [...]. Quindi la ragione, o la causa, per cui Dio, o la Natura,
	opera, e per cui esiste, è la medesima, cioè una sola. Come dunque esso non esiste per alcun fine, esso anche non opera per alcun fine; e come per il suo esistere, così per il suo operare esso non ha alcuna ragione né alcuno scopo. La causa detta finale non è nulla all’infuori dello stesso appetito umano, in quanto è considerato il principio o ragione o causa
	primaria di una cosa: quando diciamo, per esempio, che la causa finale di questa o quella
	casa è stata l’abitarci, noi sicuramente non intendiamo altro che questo, che un Uomo, per aver immaginato i vantaggi del disporre di una casa per viverci, ha avuto il desiderio, o l’appetito, di costruirsela. Quindi l’abitare, in quanto è considerato causa finale, non è altro
	che questo specifico appetito, il quale è in realtà una causa efficiente: che è considerata
	causa prima perché gli umani, ordinariamente, ignorano le cause dei loro appetiti. Essi so-
	no infatti, come ho detto spesso, ben consapevoli delle loro azioni e dei loro appetiti, ma
	ignari delle cause dalle quali essi sono determinati ad appetire qualcosa. Quel che poi si di-
	ce dalla gente, che la Natura talvolta sia manchevole, o sbagli per sbagliare, e produca cose
	imperfette, va annoverato tra le fantasie di cui ho trattato nell’Appendice della Prima Par-
	te. Quindi la "perfezione" e 1’"imperfezione" sono, in realtà, soltanto modi del pensare: vale a dire, nozioni che noi ci costruiamo col confrontare fra di loro individui della medesima
	specie o del medesimo genere: e per questa ragione ho detto più sopra che coi
	termini realtà e perfezione io intendo la medesima cosa (la sostanza, cioè la realtà è perfetta perché si è sviluppata seguendo le sue leggi necessarie, non sarebbe potuta essere in nessun altro modo). Noi siamo soliti, infatti, ridurre
	tutti gli esseri che sono in Natura a quell’unico genere che è chiamato generalissimo: appunto alla nozione di ente, ossia di cosa che è: nozione pertinente a tutte, senza eccezione,
	le cose che sono in Natura. In quanto, allora, noi riduciamo tutti gli esseri individui della
	Natura a questo genere, e li confrontiamo fra di loro, e rileviamo che taluni hanno più entità, o più realtà, o sono più cosa, di altri, in tanto noi diciamo che gli uni sono più perfetti
	degli altri; e in quanto attribuiamo a questi ultimi qualche particolarità che implica una ne-
	gazione - come limite, finitezza, impotenza eccetera - in tanto noi li definiamo imperfetti:
	questo, però, perché la nostra Mente non ne è colpita come dagli esseri che a noi sembrano
	perfetti, e non perché agli esseri "imperfetti" manchi qualcosa che ad essi compete o perché
	la Natura abbia sbagliato. Alla natura di una qualsiasi cosa non compete infatti nient’altro
	che ciò che deriva dalla necessità della natura della causa efficiente; e qualsiasi cosa che derivi dalla necessità della natura della causa efficiente viene ad essere necessariamente.
	Quanto ai termini di bene e di male, anch’essi non indicano alcunché di positivo nelle cose, se le consideriamo in sé, e non sono altro che modi del pensare, ossia nozioni, che noi ci
	formiamo in conseguenza del nostro confrontare le cose le une con le altre. Una stessa cosa, infatti, può essere nello stesso tempo buona, e cattiva, e anche indifferente: la Musica,
	per esempio, è buona per chi è melanconico e cattiva per chi soffre; e per chi è sordo non è
	buona né cattiva. Ma, sebbene le cose stiano così, ci conviene egualmente continuare ad
	usare quei termini. Poiché, infatti, noi vogliamo configurare un’idea di Uomo che sia il modello della natura umana, al quale fare poi riferimento, ci sarà utile conservare i termini in
	parola nel senso che ho detto. Di qui in poi, pertanto, intenderò per buono (o per bene) ciò
	che sappiamo con certezza essere un mezzo per avvicinarci sempre più a quel modello della
	natura umana che ci proponiamo; per cattivo (o per male) invece intenderò ciò che sappiamo con certezza esserci d’ostacolo alla realizzazione in noi di quel modello. In base a questo noi definiremo gli umani come più perfetti o più imperfetti in proporzione del loro maggiore o minore avvicinarsi al modello predetto. Si deve poi far molta attenzione a questo:
	che quando dico che un umano passa da una minore ad una maggiore perfezione io intendo dire non che quegli cambi in un’altra essenza o forma la sua propria essenza o forma
	(un cavallo, per esempio, cessa di esistere come cavallo sia che si muti in un Uomo, sia che
	si muti in un insetto): ma che noi ci rendiamo conto che la sua potenza di agire, in quanto
	essa risulta dalla sua natura, aumenta o diminuisce. Infine, per perfezione in generale intenderò, come ho detto, la realtà, cioè la natura di una cosa qualsiasi in quanto essa esiste
	ed agisce in un certo modo, senza alcun riferimento alla sua durata. Nessuna cosa singola
	può infatti dirsi più perfetta perché ha perseverato più a lungo nell’esistere, dato che la durata delle cose non può determinarsi in base alla loro essenza. L’essenza delle cose, invero,
	non implica alcuna certa e determinata durata dell’esistenza nel tempo: ma una cosa qualsiasi, sia essa più o meno perfetta, potrà sempre perseverare nell’esistenza con la medesima forza con la quale comincia ad esistere: così che in questo tutte le cose sono eguali.\footnote{Etica quarta parte prefazione.}
\end{quotation}



